%%%%%%%%%%%%%%%%%%%%%%%%%%%%%%%%%%%%%%%%%%%%%%%%%%%%%%%%%%%%%%%%%%%%%%%%%%%%%%%
% podstawowe definicje
%%%%%%%%%%%%%%%%%%%%%%%%%%%%%%%%%%%%%%%%%%%%%%%%%%%%%%%%%%%%%%%%%%%%%%%%%%%%%%%
\subsection{Czasoprzestrzeń w teorii względności}
Modelem ogólnej teorii względności jest czterowymiarowa 
rozmaitość różniczkowa. Rozmaitość tę budujemy na 
zbiorze, nazywanym czasoprzestrzenią, punktów nazywanych zdarzeniami. 
Zakładamy, że zbiór ten ma strukturę rozmaitości różniczkowej.
Niech $M$ będzie niepustą przestrzenią Hausdorffa (czyli taką, że
dla każdych dwóch punktów $p,q\in M$ 
istnieją rozłączne otoczenia $U_p,U_q$ odpowiednio punktów $p,q$.) 
Mapą w otoczeniu $U$ punktu $p\in M$ nazywamy parę $(U,\xi$, gdzie  
$\xi : U \to R^n$ jest homeomorfizmem (ciągłą bijekcją, której 
odwrotność jest ciągła). Homeomorfizm $\xi$ nazywamy układem współrzędnych 
w otoczeniu $p$.
Mówimy, że mapy dwie mapy są zgodne, jeżeli $\xi_1 \circ \xi_2$ (tam 
gdzie ma sens)
jest dyfeomorfizmem klasy $C^k$ (homeomorfizm z 
ciągłymi pochodnymi stopnia $k$)
Zbiór $A$ map parami zgodnych (o zgodności klasy $C^k$) 
takich że pokrywają cały zbiór $M$ nazymamy 
atlasem klasy $C^k$. Atlasem maksymalnym nazywamy atlas do którego
nie można dodać kolejnej mapy bez złamania zgodności.
Rozmaitością różniczkową klasy $C^k$ nazywamy 
zbiór $M$ z atlasem maksymalnym klasy $C^k$.
W szczególnej i ogólnej teorii względności przyjmujemy rozmaitość 

Mając mapę w punkcie $p$ możemy określić bazę w danym punkcie
za pomocą wektorów stycznych do linii układu współrzędnych.
Taką bazę należy rozumieć jako bazę lokalną (w punkcie $p$).


Będziemy stosować konwencję sumacyjną Einsteina.
Wzór na element liniowy $\d s^2 = g_{ij} \d x^i \d x^j$ definiuje 
tensor metryczny $g$.
Przyjmujemy konwencję w której $g$ ma sygnaturę $(+,-,-,-)$.
\begin{align}
g(u,u) > 0& \implies u \text{ - wektor czasowy}\\
g(u,u) = 0& \implies u \text{ - wektor zerowy}\\
g(u,u) < 0& \implies u \text{ - wektor przestrzenny}
\end{align}
W bazie ortonormalnej współczynniki tensor metryczny przybiera postać 
postać 
\begin{align}
( g_{\mu\nu} ) = \left(
\begin{array}{cccc}
1 & 0 & 0 & 0\\
0 & -1 & 0 & 0 \\
0 & 0 & -1 & 0 \\
0 & 0 & 0 & -1 
\end{array}
\right)
\end{align}
