%%%%%%%%%%%%%%%%%%%%%%%%%%%%%%%%%%%%%%%%%%%%%%%%%%%%%%%%%%%%%%%%%%%%%%%%%%%%%%%
% podstawowe definicje
%%%%%%%%%%%%%%%%%%%%%%%%%%%%%%%%%%%%%%%%%%%%%%%%%%%%%%%%%%%%%%%%%%%%%%%%%%%%%%%
\subsection{Wstępne pojęcia i konwencje}
Będziemy stosować konwencję sumacyjną Einsteina. Indeksy 
oznaczane literami greckimi zmieniają się w zakresie od 0 do 3, 
natomiast indeksy oznaczane literami arabskimi 
w zakresie od 1 do 3. Jednoski wybieramy tak, aby $c=1$.

Modelem ogólnej teorii względności jest czterowymiarowa Lorenzowska 
rozmaitość różniczkowa. Rozmaitość tę budujemy na 
zbiorze, nazywanym czasoprzestrzenią, punktów nazywanych zdarzeniami. 
Zakładamy, że zbiór ten ma strukturę rozmaitości różniczkowej.
Niech $M$ będzie niepustą przestrzenią Hausdorffa (czyli taką, że
dla każdych dwóch punktów $p,q\in M$ 
istnieją rozłączne otoczenia $U_p,U_q$ odpowiednio punktów $p,q$.) 
Mapą w otoczeniu $U$ punktu $p\in M$ nazywamy parę $(U,\xi$, gdzie  
$\xi : U \to R^n$ jest homeomorfizmem (ciągłą bijekcją, której 
odwrotność jest ciągła). Homeomorfizm $\xi$ nazywamy układem współrzędnych 
w otoczeniu $p$.
Mówimy, że mapy dwie mapy są zgodne, jeżeli $\xi_1 \circ \xi_2$ (tam 
gdzie ma sens)
jest dyfeomorfizmem klasy $C^k$ (homeomorfizm z 
ciągłymi pochodnymi stopnia $k$)
Zbiór $A$ map parami zgodnych (o zgodności klasy $C^k$) 
takich że pokrywają cały zbiór $M$ nazymamy 
atlasem klasy $C^k$. Atlasem maksymalnym nazywamy atlas do którego
nie można dodać kolejnej mapy bez złamania zgodności.
Rozmaitością różniczkową klasy $C^k$ nazywamy 
zbiór $M$ z atlasem maksymalnym klasy $C^k$.
Wymiarem rozmaitości nazywamy wymiar przestrzeni $R^n$, na której 
modelujemy rozmaitość. Od teraz przyjmujemy, że rozmaitość 
jest klasy $C^\infty$ oraz $n=4$.

Mając mapę w punkcie $p$ możemy określić bazę w danym punkcie
za pomocą wektorów stycznych do linii układu współrzędnych.
Taką bazę należy rozumieć jako bazę lokalną (bazę w punkcie $p$).


Rozmaitość jest Lorenzowska jeśli określona na niej tensor
metryczny $g$
 ma sygnaturę $(+,-,-,-)$.
W bazie ortonormalnej macierz tensora metrycznego
przybiera postać 
\begin{align}
( g_{\mu\nu} ) = \left(
\begin{array}{cccc}
1 & 0 & 0 & 0\\
0 & -1 & 0 & 0 \\
0 & 0 & -1 & 0 \\
0 & 0 & 0 & -1 
\end{array}
\right)
\end{align}
Tensor metryczny określa następujący podział wektorów
\begin{align}
g(u,u) > 0& \implies u \text{ - wektor czasowy}\\
g(u,u) = 0& \implies u \text{ - wektor zerowy}\\
g(u,u) < 0& \implies u \text{ - wektor przestrzenny}
\end{align}


