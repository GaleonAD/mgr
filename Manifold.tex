%%%%%%%%%%%%%%%%%%%%%%%%%%%%%%%%%%%%%%%%%%%%%%%%%%%%%%%%%%%%%%%%%%%%%%%%%%%%%%%
% podstawowe definicje
%%%%%%%%%%%%%%%%%%%%%%%%%%%%%%%%%%%%%%%%%%%%%%%%%%%%%%%%%%%%%%%%%%%%%%%%%%%%%%%
\subsection{Czasoprzestrzeń w teorii względności}
Modelem ogólnej teorii względności jest czterowymiarowa 
rozmaitość różniczkowa. Rozmaitość tę budujemy na 
zbiorze, nazywanym czasoprzestrzenią, punktów nazywanych zdarzeniami. 
Zakładamy, że zbiór ten ma strukturę rozmaitości różniczkowej to znaczy, 
że 
Przyjmujemy konwencję iloczynu skalarnego $g$ z sygnaturą $(+,-,-,-)$.
\begin{align}
g(u,u) > 0& \implies u \text{ - wektor czasowy}\\
g(u,u) = 0& \implies u \text{ - wektor zerowy}\\
g(u,u) < 0& \implies u \text{ - wektor przestrzenny}
\end{align}
