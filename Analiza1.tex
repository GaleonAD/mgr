%%%%%%%%%%%%%%%%%%%%%%%%%%%%%%%%%%%%%%%%%%%%%%%%%%%%%%%%%%%%%%%%%%%%%%%%%%%%%%%
% Rownanie na faze
%%%%%%%%%%%%%%%%%%%%%%%%%%%%%%%%%%%%%%%%%%%%%%%%%%%%%%%%%%%%%%%%%%%%%%%%%%%%%%%
\newpage
\section{Analiza równania fazy zegara}
W tej części przeprowadzimy analizę równania na fazę zegara
wyprowadzonego w poprzedniej części.
Interesującym nas parametrem jest przybliżenie właściwe, będące miarą 
przyspieszenia jakie działa na obiekt. 

\subsection{Zegar w przypadku stałego przyspieszenia}
Zakładamy stałe przyspieszenie właściwe~$\alpha$. Wtedy wektor 
przyspieszenia określony jest przez parametr~$\chi$

Załóżmy szczególną postać $\chi (s) = p s + q$, 
gdzie $p,q = const(s)$. {\color{red} Gdy $p=0$ 
rozpatrujemy przypadek gdy wektor $A$ jest podczas ruchu
transportowany za pomocą transportu~\eqref{FW}. } 
W tym przypadku rozwiązanie uzyskujemy stosując  
podstawienie~\eqref{analiza_podstawienie1}
\begin{align} \label{analiza_podstawienie1}
\Phi &= \varphi - \chi,\\
\frac{\d \Phi}{\d s} &= \frac{\d \varphi }{\d s} - p
\end{align}
\begin{align*}
\frac{\d \Phi}{\d s} &= \pm \frac{2}{\ell} - p  + 
\alpha \sin (\Phi ) 
\end{align*}
\begin{align*}
\d s & = \frac{\d \Phi}{ \pm \frac{2}{\ell} - p  + 
\alpha \sin (\Phi ) }
\end{align*}
Całkując prawą stronę powyższej równości stosujemy podstawienie
 $ x = \text{tg} (\Phi/2)$. Dla uproszczenia stosujemy oznaczenia
$B = \pm \frac{2}{\ell} - p $,
$C =  \sqrt{ 1 - \frac{\alpha^2}{B^2}}$
\begin{align*}
s +s_0 & = \frac{2}{BC} \text{arctg}  
\left( \frac{ \text{tg} (\Phi/2)}{C} +\frac{\alpha}{BC} \right),
\end{align*}
\begin{align*}
\varphi = ps + q + 
2\text{arctg} \left( 
C \text{tg} \left( BC(s + s_0)/2\right)  - \frac{\alpha}{B}
\right)
\end{align*}

Zauważmy, że dla $\alpha \to 0$ rozwiązanie jest 
postaci~\eqref{analityczne_graniczne}. To znaczy, że w przypadku ruchu
bez przyspieszeń nasz model zegara mierzy czas własny~$s$.
\begin{align}\label{analityczne_graniczne}
\varphi = \pm \frac{2}{\ell} s + const.
\end{align}

Zakładając warunek początkowy postaci $\varphi(0) = -\pi/2$, 
czyli $\Phi(0) = -\pi/2 - q$ możemy wyznaczyć stałą całkowania~$s_0$.
\begin{align*}
s_0 & = \frac{2}{BC} \text{arctg}  
\left( - \frac{1}{C}\text{tg} (q/2 + \pi/4) +\frac{\alpha}{BC} \right),
\end{align*}

\newpage
\subsection{Rozwiązanie przybliżone}
Interesuje nas jak rozwiązanie zachowuje się dla małych przyspieszeń. 
Rozwiążemy równanie~\eqref{phi_equation} stosując 
rachunek zaburzeń ze względu na 
parametr~$\alpha$. W tym celu zapisujemy~$\phi$ oraz~$\chi$ w postaci
szeregów~\eqref{phiszereg}~\eqref{chiszereg}. 
W równaniu~\eqref{phi_equation} zapisujemy sinus w postaci 
szeregu~\eqref{phi_equation_sin_szereg}. Następnie wstawiamy 
rozwinięcia~$\phi$~i~$\chi$ do uzyskanego równania i porządkujemy wyrazy
ze względu na~$\alpha$, odrzucając wyrazy $O(\alpha^2)$. 
Separujemy równanie ze względu na~$\alpha$ dostając 
równania~\eqref{phi_szereg_rownania}, 
których rozwiązania wyglądają 
następująco~\eqref{phi_szereg_rozwiazania}.
Ostatecznie szukane przez nas rozwiązanie ma 
postać~\eqref{phi_szereg_rozwiazanie}.
\begin{align}\label{phiszereg}
\varphi = \sum_{n=0}^{\infty} \alpha^n \varphi_n, \\
\chi = \sum_{n=0}^{\infty} \alpha^n \chi_n  \label{chiszereg}
\end{align}
\begin{align}\label{phi_equation_sin_szereg}
\dot{\varphi} \mp \frac{2}{\ell} - \alpha
\sum_{n=0}^{\infty} (-1)^n 
\frac{(\varphi-\chi)^{2n+1}}{(2n+1)!} =0
\end{align}
\begin{align}\label{phi_szereg_rownania}
\left\{ 
\begin{aligned}
\dot{\varphi_0} & = \pm \frac{2}{\ell} , &\quad & 
\varphi_0(0)=-\frac{\pi}{2},\\
\dot{\varphi_1} & = \sin (\varphi_0 - \chi_0  ), &\quad & 
\varphi_1(0) = 0 .
\end{aligned}
\right.
\end{align}
\begin{align}\label{phi_szereg_rozwiazania}
\left\{ 
\begin{aligned}
\varphi_0 & =  \pm \frac{2}{\ell}s - \frac{\pi}{2},\\
\varphi_1 & =  -\alpha \int_0^s \cos 
(\pm 2 s_1 / \ell  - \chi_0(s_1)  ) \d s_1 .
\end{aligned}
\right.
\end{align}
\begin{align}\label{phi_szereg_rozwiazanie}
\varphi =  \pm \frac{2}{\ell}s - \frac{\pi}{2} 
-\alpha  \int_0^s \cos (\pm 2 s_1 / \ell  - \chi_0(s_1)  ) \d s_1 
+O(\alpha^2).
\end{align}
Z rozwiązania przybliżonego~\eqref{phi_szereg_rozwiazanie} 
wiemy, że dla małych przyspieszeń nasz model zegara dobrze 
mierzy czas własny~$s$. Przyspieszenie charakterystyczne dla 
którego efekt powinien mieć istotny wpływ to~\eqref{alpha_c}.
Wpływ zaburzenia~$\varphi_1$ na działanie zegara jest 
rzędu~\eqref{rzadpoprawki}.
\begin{align}\label{alpha_c}
\alpha_c = \frac{2}{\ell}
\end{align}
\begin{align}\label{rzadpoprawki}
\epsilon = \frac{\alpha}{\alpha_c}
\end{align}

