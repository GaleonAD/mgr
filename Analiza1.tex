%%%%%%%%%%%%%%%%%%%%%%%%%%%%%%%%%%%%%%%%%%%%%%%%%%%%%%%%%%%%%%%%%%%%%%%%%%%%%%%
% Rownanie na faze
%%%%%%%%%%%%%%%%%%%%%%%%%%%%%%%%%%%%%%%%%%%%%%%%%%%%%%%%%%%%%%%%%%%%%%%%%%%%%%%
\newpage
\section{Analiza równania fazy zegara}
W tym rozdziale przeprowadzimy analizę równania na fazę zegara.
Najbardziej interesującym nas parametrem jest przyspieszenie właściwe.

\subsection{Zegar w przypadku stałego przyspieszenia.}
Jeśli założymy stałe przyspieszenie właściwe~$\alpha$, to wektor 
przyspieszenia $A$ będzie zależeć od parametru~$\chi$.
Załóżmy dodatkowo szczególną postać $\chi (s) = p s + q$, 
gdzie $p,\ q = const(s)$.  Gdy $p=0$, to wektor $A$ jest podczas ruchu
transportowany za pomocą transportu~\eqref{FW}.  
W tym przypadku rozwiązanie uzyskujemy stosując  
podstawienie
\begin{align} \label{analiza_podstawienie1}
\Phi &= \varphi - \chi,\\
\frac{\d \Phi}{\d s} &= \frac{\d \varphi }{\d s} - p,  \nonumber \\
\frac{\d \Phi}{\d s} &= \pm \frac{2}{\ell} - p  + 
\alpha \sin (\Phi ) ,  \nonumber \\
\d s & = \frac{\d \Phi}{ \pm \frac{2}{\ell} - p  + 
\alpha \sin (\Phi ) } . \nonumber
\end{align}
Całkując prawą stronę powyższej równości stosujemy podstawienie
 $ x = \text{tg} (\Phi/2)$. Dla uproszczenia wprowadzamy oznaczenia
$B = \pm \frac{2}{\ell} - p $,
$C =  \sqrt{ 1 - \frac{\alpha^2}{B^2}}$. 
\begin{align*}
s +s_0 & = \frac{2}{BC} \text{arctg}  
\left( \frac{ \text{tg} (\Phi/2)}{C} +\frac{\alpha}{BC} \right).
\end{align*}
Po oczywistych przekształceniach otrzymujemy 
\begin{align*}
\varphi = ps + q + 
2\text{arctg} \left( 
C \text{tg} \left( BC(s + s_0)/2\right)  - \frac{\alpha}{B}
\right) .
\end{align*}
Zauważmy, że dla $\alpha \to 0$ rozwiązanie jest 
postaci
\begin{align}\nonumber
\varphi = \pm \frac{2}{\ell} s + const.
\end{align}
Zakładając warunek początkowy $\varphi(0) = -\pi/2$
 możemy wyznaczyć stałą całkowania~$s_0$.
\begin{align*}
s_0 & = \frac{2}{BC} \text{arctg}  
\left( - \frac{1}{C}\text{tg} (q/2 + \pi/4) +\frac{\alpha}{BC} \right).
\end{align*}

\newpage
\subsection{Rozwiązanie przybliżone.}
Interesuje nas jak zegar zachowuje się dla małych przyspieszeń. 
W tym celu rozwiążemy równanie~\eqref{phi_equation} stosując 
rachunek zaburzeń ze względu na 
parametr~$\alpha$. Zapisujemy~$\phi$ oraz~$\chi$ w postaci
szeregów~\eqref{phiszereg}~\eqref{chiszereg}. 
Następnie w~równaniu~\eqref{phi_equation} rozwijamy sinus 
w szereg~\eqref{phi_equation_sin_szereg}. 
\begin{align}\label{phiszereg}
\varphi = \sum_{n=0}^{\infty} \alpha^n \varphi_n, \\
\chi = \sum_{n=0}^{\infty} \alpha^n \chi_n , \label{chiszereg}
\end{align}
\begin{align}\label{phi_equation_sin_szereg}
\dot{\varphi} \mp \frac{2}{\ell} - \alpha
\sum_{n=0}^{\infty} (-1)^n 
\frac{(\varphi-\chi)^{2n+1}}{(2n+1)!} =0.
\end{align}
Wstawiamy 
rozwinięcia~$\varphi$~i~$\chi$ do uzyskanego równania i porządkujemy wyrazy
ze względu na~$\alpha$ z dokładnością do~$O(\alpha^2)$.
 Otrzymujemy równania~\eqref{phi_szereg_rownania}, których 
rozwiązania są postaci~\eqref{phi_szereg_rozwiazania}.

\begin{align}\label{phi_szereg_rownania}
\left\{ 
\begin{aligned}
\dot{\varphi_0} & = \pm \frac{2}{\ell} , &\quad & 
\varphi_0(0)=-\frac{\pi}{2},\\
\dot{\varphi_1} & = \sin (\varphi_0 - \chi_0  ), &\quad & 
\varphi_1(0) = 0 .
\end{aligned}
\right.
\end{align}
\begin{align}\label{phi_szereg_rozwiazania}
\left\{ 
\begin{aligned}
\varphi_0 & =  \pm \frac{2}{\ell}s - \frac{\pi}{2},\\
\varphi_1 & =  -\alpha \int_0^s \cos 
(\pm 2 s_1 / \ell  - \chi_0(s_1)  ) \d s_1 .
\end{aligned}
\right.
\end{align}
Ostatecznie szukane przez nas rozwiązanie przybliżone ma 
postać
\begin{align}\label{phi_szereg_rozwiazanie}
\varphi =  \pm \frac{2}{\ell}s - \frac{\pi}{2} 
-\alpha  \int_0^s \cos (\pm 2 s_1 / \ell  - \chi_0(s_1)  ) \d s_1 
+O(\alpha^2).
\end{align}
Z rozwiązania przybliżonego wiemy, 
że dla małych przyspieszeń nasz model zegara dobrze 
mierzy czas własny~$s$. Przyspieszenie charakterystyczne, dla 
którego efekt powinien mieć istotny wpływ to
$\alpha_c = \frac{2}{\ell}$.
Wpływ zaburzenia~$\varphi_1$ na działanie zegara jest 
rzędu~$\epsilon = \frac{\alpha}{\alpha_c}.$
