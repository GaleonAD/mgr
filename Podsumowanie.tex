%%%%%%%%%%%%%%%%%%%%%%%%%%%%%%%%%%%%%%%%%%%%%%%%%%%%%%%%%%%%%%%%%%%%%%%%%%%%%%%
% Podsumowanie
%%%%%%%%%%%%%%%%%%%%%%%%%%%%%%%%%%%%%%%%%%%%%%%%%%%%%%%%%%%%%%%%%%%%%%%%%%%%%%%
\section{Podsumowanie}
Badanie hipotezy zegara jest ważne ze względu na jej 
fundamentalny charakter w teorii względności. 
W tej pracy wykorzystano fundamentalny relatywistyczny rotator 
do konstrukcji modelu najprostszego i zarazem podstawowego 
zegara. Uzyskany model determinuje równanie na fazę zegara $\varphi$,
która w układach bez przyspieszeń mierzy czas własny. 
W układach, w których pojawia się przyspieszenie, pojawia się
czynnik zaburzający w równaniu opisującym działanie zegara. 
Faza $\varphi$ przestaje być wtedy odpowiednia do pomiaru 
 czasu własnego. Sugeruje to, że hipoteza zegara może 
się załamywać, gdy pojawia się przyspieszenie. 
Zaburzenie fazy zegara wynikające z przyjętego modelu
 jest zauważalne dopiero dla
 ogromnych przyspieszeń.
Aktualnie przeprowadzane eksperymenty zdają się odbywać w zbyt małych 
przyspieszeniach. Może to być motywacją do rozwoju w kierunku 
budowy akceleratorów o większych przyspieszeniach.
