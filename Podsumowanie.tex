%%%%%%%%%%%%%%%%%%%%%%%%%%%%%%%%%%%%%%%%%%%%%%%%%%%%%%%%%%%%%%%%%%%%%%%%%%%%%%%
% Podsumowanie
%%%%%%%%%%%%%%%%%%%%%%%%%%%%%%%%%%%%%%%%%%%%%%%%%%%%%%%%%%%%%%%%%%%%%%%%%%%%%%%
\section{Podsumowanie}
Badanie hipotezy zegara jest ważne ze względu na jej 
podstawowy charakter w teorii względności. 
Fundamentalny rotator relatywistyczny dostarcza matematycznej 
interpretacji mechanizmu zegarowego, która pozwala na~zbudowanie 
modelu zegara idealnego. Krzywa chronometryczna stowarzyszona z~takim 
zegarem daje zależność fazy zegara $\varphi$, będącej 
wyróżnioną parametryzacją tej krzywej, od czasu własnego
rozumianego jako długość linii świata środka masy.
Motywowani modelem zegara idealnego wyprowadziliśmy równanie różniczkowe, 
które określa postać fazy $\varphi$. W~układach 
bez przyspieszeń równanie zredukowało się, a faza zegara stała się 
równa czasowi własnemu (z dokładnością do wyboru jednostek i 
chwili początkowej).
Następnie zbadaliśmy krzywą chronometryczną w różnych warunkach fizycznych,
w których występuje niezerowe przyspieszenie. Wyznaczyliśmy równanie na~fazę 
$\varphi$ w ruchu hiperbolicznym i 
ruchu po okręgu 
w przestrzeni Minkowskiego oraz
ruchu po~okręgu z~perspektywy odległych galaktyk. Zbadaliśmy
 również jak działa zegar w polu grawitacyjnym czarnej dziury.
W~układach tych pojawił się czynnik zaburzający działanie zegara,
który zależy nie tylko od wartości przyspieszenia, ale również 
od~jego kierunku.
Faza $\varphi$ przestaje być wtedy odpowiednia do pomiaru 
 czasu własnego. Sugeruje to, że hipoteza zegara może 
się załamywać, gdy pojawia się przyspieszenie. 
Zaburzenie fazy zegara jest zauważalne dopiero dla
 ogromnych przyspieszeń.
Aktualnie przeprowadzane eksperymenty odbywają są w warunkach o zbyt małych 
przyspieszeniach. Otrzymane wyniki mogą być motywacją do projektowania 
doświadczeń, w których można będzie zaobserwować 
załamanie się hipotezy zegara.
