%%%%%%%%%%%%%%%%%%%%%%%%%%%%%%%%%%%%%%%%%%%%%%%%%%%%%%%%%%%%%%%%%%%%%%%%%%%%%%%
% Model cząstki zerowej
%%%%%%%%%%%%%%%%%%%%%%%%%%%%%%%%%%%%%%%%%%%%%%%%%%%%%%%%%%%%%%%%%%%%%%%%%%%%%%%
\subsection{Model uogólniony}
W tej części będziemy rozważać układ w którym zakładamy
więz postaci $\dot{x} \cdot \dot{x} = 0$, to jest 
cząstkę poruszającą się po krzywej zerowej.
Lagrangian postaci
\begin{align*}
L = \sqrt{\dot{x} \cdot \dot{x} }
\end{align*}
nie jest odpowiedni do opisu takiej cząstki, gdyż 
wtedy pęd $P_\mu = \frac{\partial L}{\partial \dot{x}^\mu}$ 
nie jest skończony. 
W zastępstwie możemy użyć 
Lagrangianu postaci~\cite{}
\begin{align}~\label{nullparticlelagrangian}
L = w( \dot{x} \cdot \dot{x}).
\end{align}
Wtedy równanie Eulera-Lagrange dla zmiennej $w$ jest postaci
\begin{align*}
0 = \frac{\partial L}{\partial w} =  \dot{x} \cdot \dot{x}
\end{align*}
i zapewnia zerowość linii świata $x$. Wtedy pęd
kanoniczny związany z $x$  jest zachowany podczas ruchu i równy
\begin{align*}
P_\mu = 2 w \dot{x}_\mu, \quad P_\mu P^\mu = 0.
\end{align*}
Wtedy w układzie odniesienia, w 
którym $e$ jest wersorem czasowym
takim, że 
\begin{align*}
e\cdot \dot{x}=1
\end{align*} 
mamy
\begin{align*}
    p = U \dot{x},
\end{align*}
gdzie $U$ jest energią fotonu.


Uogólnienie biorące pod uwagę czątki poruszające się
po krzywych zerowych możemy zapisać w 
postaci~\cite{polchinski_1998}
\begin{align*}
L =  \frac{1}{2} \left(  
\eta^{-1} \dot{x} \cdot \dot{x} +  \eta m^2 f( \xi ) \right)  
+ \lambda ( k\cdot k )
\end{align*}
Równanie Eulera-Lagranga dla $\eta$ daje dwie możliwości
\begin{align*}
0 = \frac{\partial L }{\partial \eta} =  \frac{1}{2}
\left(m f(\xi) - \eta^{-2} ( \dot{x} \cdot \dot{x} )
  \right).
\end{align*}
Zakładając, że $\eta$ jest zależna od prędkości 
dostajemy więz
\begin{align*}
\eta = \frac{\sqrt{\dot{x} \cdot \dot{x}}}{m f(\xi)}.
\end{align*}
Wtedy lagrangian sprowadza się do postaci wyjściowej 
dla rotatora relatywistycznego~\ref{rotatorStar}.
Gdy założymy, że $\eta$ jest niezależne otrzymujemy
więzy 
\begin{align*}
\dot{x} \cdot \dot{x} = 0, \quad f(\xi) = 0
\end{align*}
Przyjęcie $m =0$ daje automatycznie spelnione warunki 
$P_\mu P^\mu =0$ oraz $W_\mu W^\mu = 0$ i sprowadza 
Lagrangian do postaci~\ref{nullparticlelagrangian}, 
czyli dla swobodnej cząstki 
poruszającej się po krzywej zerowej.
Będziemy zatem traktować $m$ jako parametr w ogólności
niezerowy, co implikuje więz $f(\xi)=0$.
Obliczamy $P_\mu$ oraz $W_\mu$
\begin{align*}
P_\mu &= \eta^{-1} \dot{x}_\mu - 
\frac{\eta m^2}{k\cdot \dot{x}} \xi f'(\xi) k_\mu,\\
\Pi_\mu &= \frac{\eta m^2}{\dot{k}\cdot \dot{k}}
\xi f'(\xi) \dot{k}_\mu, \\
\end{align*}
\begin{align*}
P_\mu &= -2 m^2 \xi f'(\xi),\\
W_\mu W^\mu &= \frac{m^4}{(\dot{k}\cdot \dot{k})^2}
f'(\xi)^2 \xi^2 
\left| 
\begin{array}{ccc}
\dot{k} \cdot \dot{k}& \dot{k} \cdot k& \dot{k} \cdot \dot{x}\\
k \cdot \dot{k}& k \cdot k &k  \cdot \dot{x}\\
\dot{x} \cdot \dot{k}& \dot{x} \cdot k &\dot{x} \cdot \dot{x}
\end{array}
\right|= 
- m^4 \ell^2 \xi f'(\xi)^2
\end{align*}
Ponownie zakładamy
że niezmienniki Casimira $P_\mu P^\mu$ i $W_\mu W^\mu$ 
są parametrami. Daje się 
je zapisać w postaci~\ref{casimir1}~i~\ref{casimir2}.
\begin{align*} 
 -2 \xi f'(\xi)
\stackrel{\ref{casimir1}}{=} 1 \stackrel{\ref{casimir2}}{=}
  4  \xi f'(\xi)^2
\end{align*}
Uzyskaliśmy w ten sposób odpowiednie więzy 
\begin{align*}
f(\xi) = 0, \ f'(\xi) =- \frac{1}{2},\ \xi = 1.
\end{align*}
\begin{align} \label{wiez}
 - \ell^2 \frac{\dot{k} \cdot \dot{k}}{ ( k \cdot \dot{x})^2 } = 1.
\end{align}

Wiezy te pozwola nam określić ruch zegara. 
Będziemy go opisywać w reperze który porusza się 
wraz z posiadaczem zegara.  
