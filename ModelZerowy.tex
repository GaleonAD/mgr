%%%%%%%%%%%%%%%%%%%%%%%%%%%%%%%%%%%%%%%%%%%%%%%%%%%%%%%%%%%%%%%%%%%%%%%%%%%%%%%
% Model cząstki zerowej
%%%%%%%%%%%%%%%%%%%%%%%%%%%%%%%%%%%%%%%%%%%%%%%%%%%%%%%%%%%%%%%%%%%%%%%%%%%%%%%
\subsection{Model singularny.}
W tej części przedstawimy pokrótce uogólnioną konstrukcję 
Lagrangianu zegara motywowaną przedstawionym
 modelem Staruszkiewicza i zawartą w~pracy~\cite{Bratek2015wiele}.
Lagrangian opisujący relatywistyczny rotator powinien być 
niezmienniczy ze względu na lokalne skalowanie 
$\delta k =\epsilon k$. Aby wariacja 
$\delta L $ znikała dla dowolnego $\epsilon$, musimy założyć, że
 $k \cdot \Pi = 0$. Pamiętając, że $k$ jest kierunkiem zerowym, 
mamy dwa więzy wynikające ze struktury rotatora
\begin{align}\label{W1}
k \cdot \Pi = 0, \quad k \cdot k = 0. \tag{W1}
\end{align}
Następnie zakładając, że rotator ma być fundamentalny, 
mamy dwa więzy nakładane na niezmienniki Casimira 
grupy Poincarégo~\ref{casimir1},\ \ref{casimir2}.
W ten sposób otrzymujemy
cztery więzy będące punktem wyjścia
do konstrukcji Lagrangianu zegara.
Więzy~\eqref{W1} pozwalają zapisać kwadrat pseudowektora $W$ w postaci
\begin{align*}
W_\mu  W^\mu = 
-\left| 
\begin{array}{ccc}
P\cdot P& P \cdot k& P \cdot \Pi\\
k\cdot P& k \cdot k& k \cdot \Pi\\
\Pi \cdot P& \Pi \cdot k& \Pi \cdot \Pi\\
\end{array}
\right| = 
(P \cdot k)^2  (\Pi \cdot \Pi). \\
\end{align*}

Dirac, w jednej ze swoich najważniejszych prac: 
"Generalized Hamiltonian Dynamics"~\cite{DiracHam}, podał
jak kon- struować Lagrangiany z więzami dla prędkości.
Zgodnie z metodą Diraca, postać Lagrangianu dla zegara wynika 
z~postaci Hamiltonianu zegara Staruszkiewicza. Ten Hamiltonian
jest liniową kombinacją więzów pierwszego rodzaju (tj. 
takich, że parami zerują nawiasy Poissona) ze współczynnikami $u_i$ 
(dla $ i=1,\ 2,\ 3,\ 4$), będącymi dowolnymi
funkcjami zmiennej niezależnej $\tau$ 
\begin{align*}
H = \frac{u_1}{2m} \left(P\cdot P -m^2 \right) 
+ \frac{u_2}{2m } \left( P\cdot P + \frac{4}{\ell^2 m^2} (k\cdot P)^2 
(\Pi \cdot \Pi) \right)
+ u_3 (k \cdot \Pi ) + u_4 (k \cdot k).
\end{align*}
Z równań $\frac{\partial H}{\partial u_i} = 0 ,\ i=1,\ 2 ,\ 3,\ 4$ 
dostajemy wyjściowy
układ więzów, natomiast prędkości dane są przez
\begin{align*}
\dot{x} &= \frac{\partial H}{\partial P} = 
\frac{u_1 + u_2}{m}P - u_2 \frac{m}{k\cdot P} k,\\
\dot{k} &= \frac{\partial H}{\partial \Pi} =  
u_2\frac{4 (k\cdot p)^2}{\ell^2m^3}\Pi +u_3 k .
\end{align*}

Transformacją
prowadzącą od Hamiltonianu do Lagrangianu jest 
przekształcenie odwrotne Legendre'a. 
W tym przypadku ma ono maksymalny rząd, gdy $\dot{x} \cdot \dot{x}$ 
jest normalizowalne. Jednak powstały w ten sposób Lagrangian, taki 
jak~\eqref{rotatorStar},
ma~defekt, bo zastosowana do niego zasada Hamiltona nie determinuje ruchu. 
Natomiast w przypadku $\dot{x} \cdot \dot{x} = 0$  rząd transformacji
 się obniża. 
Ta osobliwość odwrotnej transformacji Legendre'a 
wyróżnia zegar idealny, który ma własność, będącą klasycznym 
odpowiednikiem zjawiska Zitterbewegung
odkrytego przez Schrödingera dla elektronu Diraca
(w naszym przypadku jest to ruch kołowy z prędkością światła). 
Zastanówmy się, dlaczego ruch z prędkością światła występuje w modelu,
pomimo że mamy do czynienia z cząstką masywną.
Otóż z Hamiltonianu Diraca dla elektronu swobodnego wynika, że 
wartości własne operatora prędkości dla kwantowej cząstki
relatywistycznej są równe $\pm c$. 
Dirac zawarł w~swojej książce "The Principles of
Quantum Mechanics"~\cite{principia} 
wyjaśnienie tego fenomenu dla kwantowych
cząstek relatywistycznych. Mianowicie, że wartości pędu relatywistycznego
są nieograniczone, a co za tym idzie 
wartość oczekiwana pędu jest nieskończona. Więc 
z relatywistycznego związku pęd-prędkość otrzymujemy wartość $c$.
Pomimo tego, ruch środka masy odbywa się z prędkością mniejszą niż 
prędkość światła $c$, co też zachodzi 
dla naszego zegara.

Założenie 
$u_1 = u_2 \neq 0$ obniża rząd transformacji 
odwrotnej Lagendre'a i daje dodatkowe więzy 
\begin{align}\boxed{
\label{wiezD}
\frac{\dot{x}\cdot\dot{x}}{\dot{x}\cdot k}=0, 
}
\end{align}
\begin{align} 
\boxed{
\label{wiez}
 - \ell^2 \frac{\dot{k} \cdot \dot{k}}{ ( k \cdot \dot{x})^2 } = 1 .
}
\end{align}
Otrzymujemy Lagrangian postaci
\begin{align*}\boxed{
L  = \frac{m \kappa}{2} \frac{\dot{x}\cdot\dot{x}}{k\cdot\dot{x}}
+ \frac{m }{4 \kappa} 
 \left( \ell^2 \frac{\dot{k} \cdot \dot{k}}{  k \cdot \dot{x} } + 
k\cdot\dot{x} \right) + \lambda (k\cdot k),}
\end{align*}
gdzie $\kappa $ jest zmienną niezależną, a $\lambda$ 
mnożnikiem Lagrange'a~\cite{Bratek2015wiele}.
Z powyższego Lagrangianu i z więzów mamy
\begin{align*}
P = m \left( \frac{\kappa}{k\cdot \dot{x}} \dot{x}+ \frac{1}{2\kappa} k\right).
\end{align*}
Zatem pęd $P$ dany jest kombinacją liniową wektorów $k$ i $\dot{x}$. Przyjmując
$e=P/m$ dostajemy
\begin{align}\label{r123}
\frac{\dot{x}}{e\cdot \dot{x}} = 
2 e - \frac{k}{e\cdot k}.
\end{align}
Powyższa równość określa zależność między wektorem 
stycznym do linii świata środka masy, a wektorami~$k$~i~$\dot{x}$.
Przyjęcie $\dot{x} \cdot \dot{x} = f$, gdzie $f$ jest pewną 
funkcją $s$, nie ustala prędkości, 
i trzeba w tym celu wspomóc się jakimś konkretnym obserwatorem.
Wybór obserwatora jest równoważny z arbitralnym ustaleniem 
kąta hiperbolicznego między wektorami $\dot{x}$ 
i $e$. 
%W jednorodnej przestrzeni każdy obserwator jest 
%równouprawniony, co przekłada się na arbitralność
%wyboru fazy zegara~\cite{Bratek2015wiele}.
W jednorodnej przestrzeni czteroprędkości (Łobaczewskiego) 
każdy punkt
(określający czterowektor czasowy, a więc reprezentujący konkretnego
obserwatora) jest równouprawniony, co przekłada się na arbitralność wyboru
fazy zegara~\cite{Bratek2015wiele}.
Warunek $\dot{x} \cdot \dot{x} = 0$ determinuje ruch wewnętrzny zegara 
w przestrzeni Minkowskiego w sposób niezmienniczy wyłącznie na 
podstawie struktury stożkowej. 
Więz~\eqref{wiez} daje ograniczenie na częstość pracy zegara. 
W dalszej części pracy przy jego pomocy oraz związku~\eqref{r123} 
określimy pojęcie krzywej chronometrycznej.
