%%%%%%%%%%%%%%%%%%%%%%%%%%%%%%%%%%%%%%%%%%%%%%%%%%%%%%%%%%%%%%%%%%%%%%%%%%%%%%%
% Model cząstki zerowej
%%%%%%%%%%%%%%%%%%%%%%%%%%%%%%%%%%%%%%%%%%%%%%%%%%%%%%%%%%%%%%%%%%%%%%%%%%%%%%%
\subsection{Model uogólniony}
W tej części przedstawimy pokrótce uogólnioną konstrukcję 
Lagrangianu zegara motywowaną przedstawionym
 modelem Staruszkiewicza i zawartą w~\cite{Bratek2015wiele}.
Lagrangian opisujący relatywistyczny rotator powinien być 
niezmienniczy ze względu na lokalne skalowanie 
$\delta k =\epsilon k$. Aby wariacja 
$\delta L $ znikała dla dla dowolnego $\epsilon$ musimy założyć, że
 $k \cdot \Pi = 0$. Pamiętając, że $k$ jest kierunkiem zerowym 
mamy dwa więzy wynikające ze struktury rotatora
\begin{align}\label{W1}
k \cdot \Pi = 0, \quad k \cdot k = 0. \tag{W1}
\end{align}
Następnie, zakładając że rotator ma być fundamentalny, 
mamy dwa więzy nakładane na niezmienniki Casimira 
grupy Poincarego~\ref{casimir1},\ref{casimir2}.

 
Mamy więc cztery więzy będące punktem wyjścia
do konstrukcji Lagrangianu zegara.

Dirac, w jednej ze swoich najważniejszych prac: 
"Generalized Hamiltonian Dynamics"~\cite{DiracHam}, podał
jak konstruować Lagrangiany z więzami dla prędkości.
Zgodnie z metodą Diraca postać Lagrangianu dla zegara wynika 
z postaci Hamiltonianiu zegara Staruszkiewicza. Hamiltonian
ten jest liniową kombinacją więzów pierwszego rodzaju (t.j. 
takich, że parami zerują nawiasy Poissona) ze współczynnikami będącymi 
funkcjami $u_i$. 
\begin{align*}
H = \frac{u_1}{2m} \left(P\cdot P -m^2 \right) 
+ \frac{u_2}{2m } \left( P\cdot P + \frac{4}{\ell^2 m^2} (k\cdot P)^2 
(\Pi \cdot \Pi) \right)
+ u_3 (k \cdot \Pi ) + u_4 (k \cdot k)
\end{align*}
Z równań $\frac{\partial H}{\partial u_i} = 0 $ dostajemy wyjściowy
układ więzów. 
Transformacją
prowadzącą od Hamiltonianu do Lagrangianu jest 
przekształcenie odwrotne Legendre'a. 
W tym przypadku ma ono maksymalny rząd, gdy $\dot{x} \cdot \dot{x}$ 
jest normalizowalne. Jednak powstały Lagrangian
ma defekt i nie determinuje ruchu~\cite{Bratek2015wiele}.
W przypadku $\dot{x} \cdot \dot{x} = 0$  rząd się obniża. Założenie 
$u_1 = u_2 \neq 0$ daje dodatkowe więzy 
Ta osobliwość  odwrotnej transformacji Legendre'a 
wyróżnia zegar idealny, który ma własność
Zitterbewegung, będącą klasycznym odpowiednikiem 
odkrytego przez Schroedingera zjawiska dla elektronu Diraca. 
Z Hamiltonianu Diraca dla elektronu swobodnego wynika, że 
wartości własne operatora prędkości dla kwantowej cząstki
relatywistycznej są równe $\pm c$. 
Dirac w swej książce "The principles of
quantum mechanics" wyjaśnia poglądowo dlaczego tak musi być dla kwantowych
cząstek relatywistycznych: a mianowicie, wartości pędu relatywistycznego
są nieograniczone, dlatego wartość oczekiwana pędu jest nieskończona, co
daje ze zwykłego związku relatywistycznego pęd-prędkość wartość $c$.


%W tej części będza
%
%
%\begin{align*}
%H = u_1 ( P \cdot P  -  m^2 ) + 
%u_2 (W \cdot W + m^4\ell^2 /4) +
%+ u_3 (k \cdot \Pi)  + u_4(k \cdot k)
%\end{align*}
%
%
%układ w którym zakładamy
%więz postaci $\dot{x} \cdot \dot{x} = 0$, to jest 
%cząstkę poruszającą się po krzywej zerowej.
%Lagrangian postaci
%\begin{align*}
%L = \sqrt{\dot{x} \cdot \dot{x} }
%\end{align*}
%nie jest odpowiedni do opisu takiej cząstki, gdyż 
%wtedy pęd $P_\mu = \frac{\partial L}{\partial \dot{x}^\mu}$ 
%nie jest skończony. 
%W zastępstwie możemy użyć 
%Lagrangianu postaci
%\begin{align}~\label{nullparticlelagrangian}
%L = w( \dot{x} \cdot \dot{x}).
%\end{align}
%Wtedy równanie Eulera-Lagrange dla zmiennej $w$ jest postaci
%\begin{align*}
%0 = \frac{\partial L}{\partial w} =  \dot{x} \cdot \dot{x}
%\end{align*}
%i zapewnia zerowość linii świata $x$. Wtedy pęd
%kanoniczny związany z $x$  jest zachowany podczas ruchu i równy
%\begin{align*}
%P_\mu = 2 w \dot{x}_\mu, \quad P_\mu P^\mu = 0.
%\end{align*}
%Wtedy w układzie odniesienia, w 
%którym $e$ jest wersorem czasowym
%takim, że 
%\begin{align*}
%e\cdot \dot{x}=1
%\end{align*} 
%mamy
%\begin{align*}
%    p = U \dot{x},
%\end{align*}
%gdzie $U$ jest energią fotonu.

\begin{comment}

\begin{align*}
L =  \frac{1}{2} \left(  
\eta^{-1} \dot{x} \cdot \dot{x} +  \eta m^2 f( \xi ) \right)  
+ \lambda ( k\cdot k )
\end{align*}
Równanie Eulera-Lagrange'a dla $\eta$ daje dwie możliwości
\begin{align*}
0 = \frac{\partial L }{\partial \eta} =  \frac{1}{2}
\left(m f(\xi) - \eta^{-2} ( \dot{x} \cdot \dot{x} )
  \right).
\end{align*}
Zakładając, że $\eta$ jest zależna od prędkości 
dostajemy więz
\begin{align*}
\eta = \frac{\sqrt{\dot{x} \cdot \dot{x}}}{m f(\xi)}.
\end{align*}
Wtedy lagrangian sprowadza się do postaci wyjściowej 
dla rotatora relatywistycznego~\ref{rotatorStar}.
Gdy założymy, że $\eta$ jest niezależne otrzymujemy
więzy 
\begin{align*}
\dot{x} \cdot \dot{x} = 0, \quad f(\xi) = 0
\end{align*}
Przyjęcie $m =0$ daje automatycznie spełnione warunki 
$P_\mu P^\mu =0$ oraz $W_\mu W^\mu = 0$ i sprowadza 
Lagrangian do postaci~\ref{nullparticlelagrangian}, 
czyli dla swobodnej cząstki 
poruszającej się po krzywej zerowej.
Będziemy zatem traktować $m$ jako parametr w ogólności
niezerowy, co implikuje więz $f(\xi)=0$.
Obliczamy $P_\mu$ oraz $W_\mu$
\begin{align*}
P_\mu &= \eta^{-1} \dot{x}_\mu - 
\frac{\eta m^2}{k\cdot \dot{x}} \xi f'(\xi) k_\mu,\\
\Pi_\mu &= \frac{\eta m^2}{\dot{k}\cdot \dot{k}}
\xi f'(\xi) \dot{k}_\mu, \\
\end{align*}
\begin{align*}
P_\mu &= -2 m^2 \xi f'(\xi),\\
W_\mu W^\mu &= \frac{m^4}{(\dot{k}\cdot \dot{k})^2}
f'(\xi)^2 \xi^2 
\left| 
\begin{array}{ccc}
\dot{k} \cdot \dot{k}& \dot{k} \cdot k& \dot{k} \cdot \dot{x}\\
k \cdot \dot{k}& k \cdot k &k  \cdot \dot{x}\\
\dot{x} \cdot \dot{k}& \dot{x} \cdot k &\dot{x} \cdot \dot{x}
\end{array}
\right|= 
- m^4 \ell^2 \xi f'(\xi)^2
\end{align*}
Ponownie zakładamy
że niezmienniki Casimira $P_\mu P^\mu$ i $W_\mu W^\mu$ 
są parametrami. Daje się 
je zapisać w postaci~\ref{casimir1}~i~\ref{casimir2}.
\begin{align*} 
 -2 \xi f'(\xi)
\stackrel{\ref{casimir1}}{=} 1 \stackrel{\ref{casimir2}}{=}
  4  \xi f'(\xi)^2
\end{align*}
Uzyskaliśmy w ten sposób odpowiednie więzy 
\begin{align*}
f(\xi) = 0, \ f'(\xi) =- \frac{1}{2},\ \xi = 1.
\end{align*}
\begin{align} \label{wiez}
 - \ell^2 \frac{\dot{k} \cdot \dot{k}}{ ( k \cdot \dot{x})^2 } = 1.
\end{align}
\end{comment}
Więzy te pozwolą nam określić ruch zegara. 
Będziemy go opisywać w reperze który porusza się 
wraz z posiadaczem zegara.  
