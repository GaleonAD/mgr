%%%%%%%%%%%%%%%%%%%%%%%%%%%%%%%%%%%%%%%%%%%%%%%%%%%%%%%%%%%%%%%%%%%%%%%%%%%%%%%
% Model cząstki zerowej
%%%%%%%%%%%%%%%%%%%%%%%%%%%%%%%%%%%%%%%%%%%%%%%%%%%%%%%%%%%%%%%%%%%%%%%%%%%%%%%
\subsection{Model dla cząstki zerowej}
W tej części będziemy rozważać układ w którym zakładamy
więz postaci $\dot{x} \cdot \dot{x} = 0$.
Motywacją do takich rozważań może być urządzenie 
określane mianem "zegara świetlnego" (ang. light 
clock~\cite{Fletcher2013, West2007}) lub inaczej
"zegara geometrodynamicznego" (ang. geometrodynamic 
clock~\cite{ohanian2013gravitation})
W najprostrzym wydaiu składa się on z ustawionych 
naprzeciw siebie luster.
Odbijający się między nimi promień świetlny 
wyznacza częstość pracy zegara.
Poprzez dobranie odpowiednio dobrych luster można taki zegar
uczynić idealnym ze względu na mechanizm wewnętrzny.

Rozważmy cząstkę poruszającą się po krzywej zerowej.
Lagrangian postaci
\begin{align}
L = \sqrt{\dot{x} \cdot \dot{x} }
\end{align}
nie jest odpowiedni do opisu takiej cząstki, gdyż 
wtedy pęd $P_\mu = \frac{\partial L}{\partial \dot{x}^\mu}$ 
nie jest skończony. 
W zastępstwie możemy użyć 
Lagrangianu postaci~\cite{}
\begin{align}
L = w\dot{x} \cdot \dot{x}.
\end{align}
Wtedy równanie Eulera-Lagrange dla zmiennej $w$ jest postaci
\begin{align}
0 = \frac{\partial L}{\partial w} =  \dot{x} \cdot \dot{x}
\end{align}
i zapewnia zerowość linii świata $x$. Wtedy pęd
kanoniczny związany z $x$  jest zachowany podczas ruchu i równy
\begin{align}
P_\mu = 2 w \dot{x}_\mu, \quad P_\mu P^\mu = 0.
\end{align}
Wtedy w układzie odniesienia, w 
którym $e$ jest wersorem czasowym
takim, że 
\begin{align}
e\cdot \dot{x}=1
\end{align} 
mamy
\begin{align}
    p = U \dot{x},
\end{align}
gdzie $U$ jest energią.


Uogólnienie biorące pod uwagę czątki poruszające się
po krzywych zerowych możemy zapisać w 
postaci~\cite{polchinski_1998}
\begin{align}
L =  \frac{1}{2} \left(  
\eta \dot{x} \cdot \dot{x} +  \eta^{-1} f( \xi ) \right)  
+ \lambda ( k\cdot k )
\end{align}
Równanie Eulera-Lagranga dla $\eta$ daje dwie możliwości
\begin{align}
0 = \frac{\partial L }{\partial \eta} =  \frac{1}{2}
\left( \dot{x} \cdot \dot{x} -
\eta^{-2} f(\xi)  \right).
\end{align}
Zakładając, że $\eta$ jest zależna od prędkości 
dostajemy więz
\begin{align*}
\eta = \frac{f(\xi)}{\sqrt{\dot{x} \cdot \dot{x}}}.
\end{align*}
Wtedy lagrangian sprowadza się do postaci
(z dokładnością do stałej).
Gdy założymy, że $\eta$ jest niezależne otrzymujemy
więzy 
\begin{align}
\dot{x} \cdot \dot{x} = 0, \quad f(\xi) = 0
\end{align}





Przyjęcie m =0 daje automatycznie spelnione warunki 
PP =0 oraz WW=0 dla dowolnej ograniczonej nieosobliwej 
funkcji f(xi). Będziemy jednak traktować m jako parametr 
modelu co w ogóln ości daje m not 0. Wtedy zakładamy, 
że niezmienniki Casimira daje się zapisać jako PP=m2 (C1) 
oraz WW=-m4l2/2. (C2)
Obliczamy P oraz W

 Dostajemy warunki na f w postaci
4xi fpxi2 =1=
-2 xi fxi

Układ ten nadaje warunki na fp i xi postaci

Fp=-1/2
Xi=1

Wiezy te pozwola nam określić ruch zegara. 
Będziemy go opisywać w reperze który porusza się 
wraz z posiadaczem zegara. 
Wprowadzimy go w takiej postaci, aby 
