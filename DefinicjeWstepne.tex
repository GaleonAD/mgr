%%%%%%%%%%%%%%%%%%%%%%%%%%%%%%%%%%%%%%%%%%%%%%%%%%%%%%%%%%%%%%%%%%%%%%%%%%%%%%%
% podstawowe definicje
%%%%%%%%%%%%%%%%%%%%%%%%%%%%%%%%%%%%%%%%%%%%%%%%%%%%%%%%%%%%%%%%%%%%%%%%%%%%%%%
\section{Podstawowe pojęcia}
Modelem ogólnej teorii względności jest czterowymiarowa r
ozmaitość różniczkowa $M$, którą 
nazywamy czasoprzestrzenią.
\begin{definition}
Krzywą $y$ nazywamy dowolny ciągły obraz odcinka na $M$. 
Funkcję $ (a,b) \ni s \to y(s) \in M$ której obrazem jest 
krzywa $y$ nazywamy parametryzacją krzywej $y$.
\end{definition}
\begin{definition}
Wektorem stycznym do krzywej $y$ w $p$ nazywamy wektor $u$ taki, że
\begin{align*}
u = \frac{\d y(s)}{\d s}_{\big|_p},
\end{align*}
\end{definition}
Przyjmujemy konwencję iloczynu skalarnego $g$ z sygnaturą $(+,-,-,-)$.
\begin{align}
g(u,u) > 0& \implies u \text{ - wektor czasowy}\\
g(u,u) = 0& \implies u \text{ - wektor zerowy}\\
g(u,u) < 0& \implies u \text{ - wektor przestrzenny}
\end{align}
W ten sposób przestrzeń wektorów stycznych dzieli się na trzy 
klasy. Ten podział wyróżnia trzy rodzaje krzywych. 
\begin{definition}
Krzywą $\gamma$ nazywamy krzywą czasową (zerową, przestrzenną)
jeżeli każdy wektor styczny do $\gamma$ jest wektorem czasowym
(zerowym, przestrzennym).
\end{definition}
\begin{definition}
Linią świata cząstki nazywamy dowolną gładką krzywą czasową.
\end{definition}
Oczywiście dana krzywa może mieć różne parametryzacje.
Model fizyczny powinien być niezmienniczy ze względu na 
reparametryzację krzywej. Jeśli 
dwa różne parametry $s,t$ opisujace krzywą $y$
przez parametryzacje $s \to y_1(s)$ i $t\to y_2(t)$
powiązane są zależnością $t = f(s)$ to
$y_2(t) = y_1( f^{-1}(t) )$. Wtedy niezmienniczość 
działania $S$ ze względu na reparametryzację 
oznacza, że $S(y_1) = S(y_2)$.
\begin{align}
S(y_1) = \int L( y_1(s) , y'_1(s) ) \d s \\
S(y_2) = \int L ( y_2(t), y_2'(t) )  \d t = 
\int L ( y_2(f(s)), y_2'(f(s)) / f'(s) )  \d f(s) = 
\int L ( y_1(s), y_1'(s) /f'(s) ) f'(s) \d s  
\end{align} 
\begin{align}
 L( y, f \dot{y} ) = f L ( y, \dot{y}  ) 
\end{align} 
Inaczej można ten warunek zapisać, jako rządanie by
podmiana $s \to f(s)$ nie zmieniała postaci Lagrangianu.
\begin{definition}
Pochodną kowariantną nazywamy ...
transport wektora wzdłuż krzywej $\gamma$ dla którego 
.... nazywamy transportem równoległym.
\end{definition}
\begin{definition}
Pochodną absolutną nazywamy pochodną kowariantną w 
kierunku wektora stycznego $u$ parametryzowanego 
czasem własnym $s$
transport wektora wzdłuż krzywej $\gamma$ dla którego 
.... nazywamy transportem równoległym.
\end{definition}



