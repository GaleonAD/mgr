\subsection{Rozwiązanie układu równań z 
funkcją $f$ w modelu Staruszkiewicza.}
\noindent
Dany jest układ równań postaci 
\begin{align*} 
 f(\xi)^2- 4 f(\xi) f'(\xi) \xi
 =  1 =
  16   f(\xi)^2 f'(\xi)^2 \xi .
\end{align*}
Zauważmy przy tym, że 
\begin{align*}
\xi = - \ell^2 \frac{\dot{k} \cdot \dot{k}}{ ( k \cdot \dot{x})^2 } > 0.
\end{align*}
Rozwiązemy teraz pierwsze z równań, to jest
\begin{align*} 
 f(\xi)^2- 4 f(\xi) f'(\xi) \xi =  1 .
\end{align*}
Można je przekształcić do postaci równania o zmiennych 
rozdzielonych
\begin{align*} 
\frac{1}{\xi} = \frac{ 4 f(\xi) }{ f(\xi)^2-1} f'(\xi) 
\end{align*}
\begin{align*} 
\text{ln} \xi = \int \frac{ 4 f }{ f^2-1} \d f 
\end{align*}
\begin{align*} 
\text{ln} \xi = \int \frac{ 2 }{ f^2-1} \d f^2 
\end{align*}
\begin{align*} 
\text{ln} C_1 \xi = 2\text{ln}|f^2-1|, \quad C_1>0
\end{align*}
\begin{align*} 
 C_1 \xi  = (f^2-1)^2 
\end{align*}
\begin{align*} 
 \sqrt{ C_1 \xi}  = | f^2-1 | 
\end{align*}
\begin{align*} 
|f(\xi)| =  \sqrt{ 1 \pm C\sqrt{ \xi} }. 
\end{align*}
Rozwiążemy teraz równanie
\begin{align*} 
  16   f(\xi)^2 f'(\xi)^2 \xi  = 1
\end{align*}
Ponownie równanie to da się zapisać jako równanie o 
zmiennych rozdzielonych
\begin{align*} 
  4  f(\xi) f'(\xi)   =\pm \frac{1}{\sqrt{\xi}}
\end{align*}
\begin{align*} 
  2 f^2   =\pm 2 \sqrt{\xi} + 2 C_2
\end{align*}
\begin{align*} 
   f^2   =\pm  \sqrt{\xi} + C_2
\end{align*}
\begin{align*} 
   |f|   =\sqrt{ \pm  \sqrt{\xi} + C_2 }.
\end{align*}
Oba rozwiązania uzgadniamy wybierając 
stałe całkowania $C_1 = 1$ oraz $C_2=1$.



