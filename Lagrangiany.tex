%%%%%%%%%%%%%%%%%%%%%%%%%%%%%%%%%%%%%%%%%%%%%%%%%%%%%%%%%%%%%%%%%%%%%%%%%%%%%%%
% podstawowe definicje
%%%%%%%%%%%%%%%%%%%%%%%%%%%%%%%%%%%%%%%%%%%%%%%%%%%%%%%%%%%%%%%%%%%%%%%%%%%%%%%
\subsection{Niezmienniczość Lagrangianu}
Model fizyczny czątski poruszającej się po linii świata
powinien być niezmienniczy ze względu na 
reparametryzację krzywej. Niech $\tau_1,\tau_2$ będą 
dwoma różnymi parametrami opisujacymi krzywą $y$, oraz 
$f$ będzie reparametryzacją krzywej $y$ 
taką, że $\tau_2 = f(\tau_2)$.
\begin{align}
aa
\end{align} 
\begin{align}
 L( y, f \dot{y} ) = f L ( y, \dot{y}  ) 
\end{align} 
Inaczej można ten warunek zapisać, jako rządanie by
podmiana $s \to f(s)$ nie zmieniała postaci Lagrangianu.

