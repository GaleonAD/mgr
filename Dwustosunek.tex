%%%%%%%%%%%%%%%%%%%%%%%%%%%%%%%%%%%%%%%%%%%%%%%%%%%%%%%%%%%%%%%%%%%%%%%%%%%%%%%
% Rownanie na faze
%%%%%%%%%%%%%%%%%%%%%%%%%%%%%%%%%%%%%%%%%%%%%%%%%%%%%%%%%%%%%%%%%%%%%%%%%%%%%%%
\subsection{Czwórka symetryczna kierunków zerowych}
Będziemy od teraz zakładać, że jeden z wersorów bazy $E$ 
($e_3$) jest prostopadły do hiperpłaszczyzny ruchu, tak, że
\begin{align*}
A\cdot e_3 = 0.
\end{align*}
%Zawsze możemy wybrać bazę $E$ tak, aby powyższy warunek był 
%spełniony~(zob. dodatek~\ref{e3proof}). 
Nie jest to duże ograniczenie i, jak się później przekonamy, pozwala 
na zastosowanie modelu w wielu przypadkach.
Zauważmy, że wektor~$A$~leży wtedy w płaszczyźnie rozpinanej przez 
wektory~$e_1$~i~$e_2$. Licząc przyspieszenie właściwe dostajemy
\begin{align*}
\alpha^2 = (A\cdot e_1)^2 + (A\cdot e_2)^2
\end{align*}
Interpretując powyższą równość jako trójkę pitagorejską 
możemy wprowadzić następujące oznaczenia
\begin{align*}
\cos\chi = \frac{A\cdot e_1}{\alpha},\\
\sin\chi = \frac{A\cdot e_2}{\alpha}.
\end{align*}

Z wersorów $e$ i $e_3$ tworzymy dwa zerowe wektory skierowane 
w przyszłość $k_+$ i $k_-$, 
które uważamy za wektory własne pewnej transformacji Lorentza. 
\begin{align}
k_+ &= \frac{e+e_3}{\sqrt{2}} \\
k_- &= \frac{e-e_3}{\sqrt{2}} 
\end{align}
\begin{align*}
k_+\cdot k_- = 1 \qquad k_\pm\cdot k_\pm  = 0.
\end{align*}
\begin{align*}
\mathcal{O}( k_\pm ) = \frac{1}{\sqrt{2}} \mathcal{O} 
(e\pm e_3) =\frac{1}{\sqrt{2}} 
(\mathcal{O} e\pm \mathcal{O} e_3) =
\frac{1}{\sqrt{2}}  (e\pm e_3) = k_\pm.
\end{align*}
Wektory te są wektorami własnymi pewnego obrotu $\mathcal{O}$.
 Łatwo sprawdzić, że 
jest to obrót w płaszczyźnie wyznaczonej przez wersory~$e_1$ i $e_2$,
czyli eliptyczne przekształcenie Lorentza. 
Obrót ten pozwala nam zinterpretować kąt~$\chi$. 
Zauważmy, że możemy za pomocą obrotu~$\mathcal{O}$ obrócić, 
wersor wektora przyspieszenia~o kąt~$-\chi$, 
tak aby spełniał prawo transportu~\eqref{FW}. 
Schematycznie przedstawiono to na rysunku~\ref{A_chi_plot}.
\begin{figure}
\centering
\includegraphics[scale=0.6]{A_chi.eps}
\caption{Schemat obrazujący obrót~$\mathcal{O}$ wykonany na wersorze 
przyspieszenia $A/\alpha$ w bazie $E$.}{\label{A_chi_plot}}
\end{figure}

Rozważamy trzeci wektor zerowy skierowany w przyszłość $k$ taki, 
że $k\cdot e_3 \equiv 0$ oraz $k(0)\cdot e_1(0) =0$.  
%Z pierwszego warunku wynika że ostatnia współrzędna wektora $k$ 
%jest zerowa. $e_1(0)=(0,1,0,0)$ co oznacza, że 
%wektor $k(0)$ ma drugą współrzędną równą zero. 
Wektor ten rozkładamy w bazie $E$
\begin{align*}
k = k^0 e +  k^i e_i, \qquad k^1(0)=0, k^3 = 0
\end{align*}
\begin{align*}
k(0) = k^0(0) e(0) + k^2(0) e_2(0) 
\end{align*}
Rozkładając $k$ w bazie $E$ stwierdzamy, że jego współrzędne 
formują trójkę pitagorejską 
\begin{align}
(k \cdot e)^2 = (k \cdot e_1)^2 +  (k \cdot e_2)^2
\end{align}
Wprowadzamy \textbf{fazę zegara} $\varphi$ 
równością~\eqref{phi_definition}
\begin{align}\label{phi_definition}
\cos\varphi = \frac{k\cdot e_1}{k\cdot e} 
\end{align}
\begin{align*}
k = (k\cdot e) (e -\cos\varphi e_1 - \sin\varphi e_2 )
\end{align*}
%gdzie $k^0=k\cdot e$, $k^1=-k\cdot e_1$, $k^2=-k\cdot e_2$.
Z wektora $k(0)$ tworzymy wektor zerowy $k_0(s)$ tak aby spełniał 
prawo transportu~\eqref{FW}. Wiemy, że wtedy jego współrzędne~w 
bazie~$E$ są stałe. Wektor~$k_0$ ustalamy więc jako~\eqref{eqk0}.
Warunek początkowy na fazę $\varphi$ ustalamy 
na~\eqref{phi_0condition}, aby dla $s=0$ wektory
$k$ i $k_0$ reprezentowały ten sam kierunek zerowy.
\begin{align}\label{phi_0condition}
\varphi(0)=-\frac{\pi}{2}
\end{align}
\begin{align}\label{eqk0}
k_0(s) =  \sqrt{2} (e + e_2).
\end{align}
%Z warunku uzyskujemy równość 
%\begin{align*}
%\frac{\d k_0(s)\cdot e }{\d s} = 0.
%\end{align*}
%Rozpiszemy drugi warunek
%\begin{align*}
%k_0(s)_\perp = -(k_0(s)\cdot e_2)e_2
%\end{align*}
%\begin{align*}
%\frac{\d -(k_0(s)\cdot e_2)e_2 }{\d s}
% =-\frac{\d (k_0(s)\cdot e_2)}{\d s}e_2 - (k_0(s)\cdot e_2)\dot{e_2} 
%\end{align*}
%\begin{align*}
%\left( \frac{\d -(k_0(s)\cdot e_2)e_2 }{\d s}\right)_\perp =
%-\frac{\d (k_0(s)\cdot e_2)}{\d s}\left(e_2 \right)_\perp 
%-(k_0(s)\cdot e_2)\left(\dot{e_2} \right)_\perp 
%\end{align*}
%Z konstrukcji bazy mamy
%\begin{align*}
%\left(\dot{e_2} \right)_\perp  = 0, \qquad \left(e_2 \right)_\perp  = e_2
%\end{align*}
%co daje zerowanie równość
%\begin{align*}
%\frac{\d k_0(s)\cdot e_2 }{\d s} = 0.
%\end{align*}
%Z powyższych warunków widzimy, że współrzędne wektora 
%$k_0(0)$ w bazie $Z$ muszą być stałe. 
%Uwzględniając, że wektor ma być zerowy wybieramy następujący wektor
%\begin{align}
%k_0(s) = e + e_2 =  (1,0,1,0)_Z \tag{k0} \label{eq:k0}
%\end{align}
Każdemu kierunkowi zerowemu możemy przyporządkować punkt na sferze,
a następnie każdemu punktowi sfery możemy przyporządkować,
przez rzut stereograficzny, 
punkt z płaszczyzny zespolonej 
(odpowiednio uzwarconej)~\cite{star1993algebra}.
Skonstruujemy teraz czwarty wektor zerowy $k_3$, który razem z 
wektorami $k_+$, $k_0$, $k_-$ utworzy czwórkę symetryczną.
Mówimy, że wektory zerowe tworzą czwórkę symetryczną,
 kiedy dwustosunek odpowiadających im liczb zespolonych 
 wynosi $e^{\pm i\pi/3}$.  
Dwustosunek liczb zespolonych $z_0,\ z_1,\ z_2,\ z_3$ przyjmujemy w 
postaci~\eqref{dwustosunek_definicja}~\cite{star1993algebra}.
Liczby zespolone odpowiadające wektorom własnym $k_\nu$ oznaczamy
przez $\kappa_\nu$ gdzie $\nu \in \{+,\ 0,\ -,\ 3\}$.
W zależności od kolejności wektorów i 
przyjętego znaku w~\eqref{dwustosunek_kappa}
otrzymujemy dwie liczby $\kappa_3$ różniące się znakiem części 
rzeczywistej \eqref{kappa3_wynikPM}. 
Wektorowi zerowemu~$k$ odpowiada liczba $\kappa_\varphi$~\eqref{kappa}.
\begin{align}\label{kappa}
\kappa = -\cos\varphi - i \sin \varphi
\end{align}
Na rysunkach~\ref{dwustosunek_plaszczyzna} oraz~\ref{dwustosunek_sfera} 
prezentujemy wzajemne położenie 
uzyskanej czwórki symetrycznej (dla Re$ (\kappa_3) >0$) 
oraz obrazu wektora $k$. 
Uzyskane wektory wektory są liniowo niezależne i tworzą bazę 
kierunków zerowych, która dodatkowo spełnia 
prawa transportu~\eqref{FW}.
\begin{align}\label{dwustosunek_definicja}
(z_0z_1z_2z_3) = 
\frac{(z_0-z_1)}{(z_0-z_3)} 
\frac{(z_2-z_3)}{(z_2 -z_1)} .
\end{align}
\begin{align}\label{dwustosunek_kappa}
\kappa_0 = i,\ \kappa_+ = 0,\ \kappa_-=\infty,
\qquad (\kappa_0\kappa_+\kappa_-\kappa_3)
 = e^{\pm i\pi/3} \\ \label{kappa3_wynikPM}
\kappa_3 =\pm  \frac{\sqrt{3}}{2} + \frac{i}{2}, \qquad
k_3 = \sqrt{2} e\pm \frac{\sqrt{3}}{\sqrt{2}}e_1+ \frac{1}{\sqrt{2}} e_2
\end{align}
\begin{align*}
k_\mu \cdot k_\nu = 1 , 
\quad k_\nu \cdot k_\nu = 0,\quad \mu \neq \nu ,\ 
\mu,\nu \in \{0,\ +,\ -,\ 3\}
\end{align*}
\begin{figure}
\centering
\includegraphics[]{dwustosunek_plaszczyzna.eps}
\caption{Obraz czwórki symetrycznej oraz kierunku $k$ na płaszczyźnie 
zespolonej.
Punkt $\kappa_- $ utożsamiamy z punktem $\infty$.
Punkt $\kappa_\phi$ porusza się po zaznaczonym okręgu jednostkowym 
wraz ze wzrostem $\phi$. Wektorem stycznym do okręgu zaznaczono 
kierunek ruchu.}
\label{dwustosunek_plaszczyzna}
\end{figure}
\begin{figure}
\centering
\includegraphics[scale=0.8]{dwustosunek_sfera.eps}
\caption{Obraz kierunków zerowych $k_0,\ k_+,\ k_-$ wraz z kierunkiem $k$
na sferze jednostkowej. 
Punkt $\kappa_\phi$ porusza się po zaznaczonym okręgu jednostkowym 
wraz ze wzrostem $\phi$. Wektorem stycznym do okręgu zaznaczono kierunek 
ruchu.
Płaszczyzna zawierająca okrąg jest prostopadła do prostej zawierającej 
$\kappa_+$ i $\kappa_-$}
\label{dwustosunek_sfera}
\end{figure}
\newpage

