% =====  STRONA TYTUŁOWA PRACY INŻYNIERSKIEJ ====
\pagestyle{empty}
%% ------------------------ NAGŁÓWEK STRONY ---------------------------------
\includegraphics[height=37.5mm]{agh_znk_wbr_cmyk.eps}\\
\rule{30mm}{0pt}
{\large\textsf{Wydział Fizyki i Informatyki Stosowanej}}\\
\rule{\textwidth}{3pt}\\
\rule[2ex]
{\textwidth}{1pt}\\
\vspace{7ex}
\begin{center}
{\bf\LARGE\textsf{Praca magisterska}}\\
\vspace{13ex}
% --------------------------- IMIE I NAZWISKO -------------------------------
{\bf\Large\textsf{Paweł Rzońca}}\\
\vspace{3ex}
{\sf \small kierunek studiów:} {\bf\small\textsf{Fizyka Techniczna}}\\
\vspace{15ex}
%% ------------------------ DANE PRACY --------------------------------------
{\bf\huge\textsf{Badanie krzywych chronometrycznych w kontekście hipotezy zegara}}\\
\vspace{14ex}
{\sf \Large Opiekun:} {\bf\Large\textsf{dr hab. Łukasz Bratek}}\\
\vspace{22ex}
\textsf{\bf\large\textsf{Kraków, sierpień 2018}}
\end{center}
%% =====  STRONA TYTUŁOWA PRACY INŻYNIERSKIEJ  ====
\newpage
%% =====  TYŁ STRONY TYTUŁOWEJ PRACY INŻYNIERSKIEJ  ====
{\sf Oświadczam, świadomy odpowiedzialności karnej za poświadczenie nieprawdy,
że niniejszą pracę dyplomową wykonałem osobiście i samodzielnie i nie korzystałem
ze źródeł innych niż wymienione w pracy.}

\vspace{14ex}

\begin{flushright}
................................................................. \\
{\sf (czytelny podpis)}
\end{flushright}
%% =====  TYL STRONY TYTUŁOWEJ PRACY INŻYNIERSKIEJ  ====
\newpage
%% =====  TYL STRONY TYTULOWEJ PRACY MAGISTERSKIEJKIEJ ====

\newpage
\rightline{Kraków, 4 sierpnia 2018}
\begin{center}
{\bf Tematyka pracy magisterskiej i praktyki dyplomowej
Jana Nowaka,
studenta V roku studiów kierunku fizyka techniczna}\\
\end{center}

Temat pracy magisterskiej:
{\bf Badanie krzywych chronometrycznych w kontekście hipotezy zegara}\\

\begin{tabular}{rl}

Opiekun pracy:                  & dr hab. Łukasz Bratek\\
Recenzenci pracy:               & \dots \dots \dots \dots \dots \dots \dots \dots\\
Miejsce praktyki dyplomowej:    & WFMI PK, Kraków\\
\end{tabular}

\begin{center}
{\bf Program pracy magisterskiej i praktyki dyplomowej}
\end{center}

\begin{enumerate}
\item Omówienie realizacji pracy magisterskiej z opiekunem.
\item Zebranie i opracowanie literatury dotyczącej tematu pracy.
\item Praktyka dyplomowa:
\begin{itemize}
\item zapoznanie się z literaturą przedmiotu dotyczącą hipotezy zegara,
\item opracowanie podstawowych pojęć i narzędzi związanych z tematyką przedmiotu,
\item konstrukcja modelu idealnego zegara dla ciała w ruchu po okręgu,
\item powtórzenie konstrukcji modelu idealnego zegara dla metryki FLRW,
\item stworzenie skryptów do numerycznego sprawdzenia obliczeń oraz generacji wykresów.
\item sporządzenie sprawozdania z praktyk
\end{itemize}
\item Kontynuacja obliczeń związanych z tematem pracy magisterskiej.
\item Zebranie i opracowanie wyników obliczeń.
\item Analiza wyników obliczeń.
\item Opracowanie redakcyjne pracy.
\end{enumerate}


\noindent
Termin oddania w dziekanacie: \dots \dots \dots\\[1cm]

\begin{center}
\begin{tabular}{lcr}
\dots\dots\dots\dots\dots\dots\dots\dots\dots\dots\dots\dots\dots\dots\dots & ~~~ &
\dots\dots\dots\dots\dots\dots\dots\dots\dots\dots\dots\dots\dots\dots\dots \\
(podpis kierownika katedry) & & (podpis opiekuna) \\
\end{tabular}
\end{center}

\newpage

\noindent
Na kolejnych dwóch stronach proszę dołączy„ kolejno recenzje pracy popełnione przez Opiekuna oraz Recenzenta 
(wydrukowane z systemu MISIO i podpisane przez odpowiednio Opiekuna i Recenzenta pracy). 
Papierową wersję pracy (zawierającą podpisane recenzje) proszę złożyć
 w dziekanacie celem rejestracji co najmniej na tydzień przed planowaną obroną.

\newpage
Tu zostanie umieszczona recenzja opiekuna
\newpage
Tu zostanie umieszczona recenzja recenzenta
\newpage

\linespread{1.3}
\selectfont
