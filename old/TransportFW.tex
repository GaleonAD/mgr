%%%%%%%%%%%%%%%%%%%%%%%%%%%%%%%%%%%%%%%%%%%%%%%%%%%%%%%%%%%%%%%%%%%%%%%%%%%%%%%
% Transport FW
%%%%%%%%%%%%%%%%%%%%%%%%%%%%%%%%%%%%%%%%%%%%%%%%%%%%%%%%%%%%%%%%%%%%%%%%%%%%%%%
\subsection{Transport Fermiego-Walkera}
Konstrukcję zegara przeprowadzimy w lokalnie nierotującej bazie. W tej części
pracy przedstawimy koncepcje potrzebne do konstrukcji takiej bazy. Dokładne 
omówienie prezentowanych zagadnień można znaleźć np. 
tu~\cite{synge1960, FWframesconstruct}
\begin{definition}Mówimy, że wektor $v$ spełnia prawo \textbf{transportu 
Fermiego-Walkera} (FW) wzdłuż linii świata $y$ jeżeli
\begin{align} 
\frac{\widetilde{\D} v}{\d s}  : = \frac{\D v}{\d s} +
(v\cdot A) u - (v\cdot u) A = 0, \tag{FW} \label{FW}
\end{align}
gdzie $u = \dot{y}$ oraz $A=\frac{\D u}{\d s}$ to odpowiednio czterowektory
prędkości i przyspieszenia stoważyszone z~linią świata~$y$.
Wyrażenie $ \frac{\widetilde{\D} }{\d s}$ nazywamy \textbf{pochodną
Fermiego-Walkera}. Zauważmy, że w~przypadku zerowego przyspieszenia $(A=0)$
pochodna Fermiego-Walkera sprowadza się do pochodnej absolutnej, więc 
dla linii będące geodezyjną transport~\eqref{FW} sprowadza się się do 
transportu równoległego.
%Transport równoległy wektora wzdłuż linii świata jest realizowany przez
%zerowanie się pochodnej absolutnej. 
Niestety w ogólności transport równoległy
nie przekształca wektorów stycznych w wektory styczne. Transport taki 
 realizuje się przez prawa transportu~\eqref{FW}. 
%Z tego powodu jest on w niektórych sytuacjach bardziej użyteczny. 
%Możemy więc myśleć o~pochodnej Fermiego-Walkera jako
%o~uogólnieniu pochodnej absolutnej na ruchy z~niezerowym 
%przyspieszeniem~[źródło]
%Wprowadzając takie oznacznie dostajemy warunek 
%transportu FW jako zerowanie się pochodnej Fermiego-Walkera.
\end{definition}
Łatwo pokazać, że warunek~\eqref{FW} możemy zapisać w równoważnej postaci 
(zob. dodatek~A).
\begin{align}
\frac{\d (v \cdot u)}{\d s}=0, \tag{FW1} \label{FW1}
\end{align}
\begin{align}
\left(\frac{\D (v_\perp)}{\d s}\right)_\perp = 0, 
\text{ gdzie } v_\perp = v-(v\cdot u)u \tag{FW2} \label{FW2}
\end{align}


Równoważna postać transportu~\eqref{FW}, to znaczy 
warunki~\eqref{FW1} i~\eqref{FW2}, pozwala w łatwy sposób 
skonstuować bazę, której wersory będą w owy sposób transportowane 
wzdłuż pewnej linii świata.
Za wersor czasowy takiej bazy możemy zawsze obrać czterowektor prędkości 
\begin{align}
e = u = \frac{\d y}{\d s}.
\end{align}
Dobieramy do niego wzajemnie ortogonalne i odpowiednio unormowane wersory 
przestrzenne $e_i$, $i=1,2,3$. Wersory takiej bazy, z racji otrogonalności,
spełniają warunek~\eqref{FW1}. Wersor $e$ spełnia również
warunek~\eqref{FW2}, gdyż
\begin{align}
e_\perp = e - e = 0.
\end{align}
Dla wersorów przestrzennych zachodzi
\begin{align}
{e_i}_\perp = e_i - (e_i \cdot e )e = e_i.
\end{align}
Wstawiając powyższą równość do warunku~\eqref{FW2} otrzymujemy 
równość~\eqref{FWwkw}. 
%Zatem warunkiem wystarczającym, aby baza ortonormalna taka, że
%$e=u$ spełniała prawo transportu FW jest równość
\begin{align}\label{FWwkw}
\frac{\D e_i }{\d s} = 
\left( \left(\frac{\D e_i }{\d s}\right) \cdot e\right) e, 
\end{align}
Powyższą bazę, tj. $\left\{ e,e_1,e_2,e_3\right\}$ 
będziemy oznaczać przez~$E$.
Przydatną własnością bazy~$E$ jest, że dany wektor 
ma w tej bazie stałe współrzędne wtedy i tylko 
spełnia prawo transportu~\eqref{FW}.% (zob. dodatek~\ref{appendix_FW})
Aby to pokazać wystarczy rozłożyć dany wektor w bazie 
$E$ i skorzystać z 
warunków~\eqref{FW1}~i~\eqref{FW2}.

\begin{definition}
\textbf{Reperem lokalnie nierotującym} nazywamy reper ruchomy poruszający
się wraz z ciałem wzdłuż jego linii świata, którego 
wersor czasowy jest styczny do linii świata (co odpowiada czteroprędkości)
i którego wersory spełniają prawo transportu~\eqref{FW}.
\end{definition}
Reper lokalnie nierotujący jest szczególnie dogodny do opisu zjawisk 
fizycznych. W granicy nierelatywistycznej odpowiada on Newtonowskiej 
koncepcji nierotującego reperu~\cite{synge1960}.
