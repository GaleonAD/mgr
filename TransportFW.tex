%%%%%%%%%%%%%%%%%%%%%%%%%%%%%%%%%%%%%%%%%%%%%%%%%%%%%%%%%%%%%%%%%%%%%%%%%%%%%%%
% Transport FW
%%%%%%%%%%%%%%%%%%%%%%%%%%%%%%%%%%%%%%%%%%%%%%%%%%%%%%%%%%%%%%%%%%%%%%%%%%%%%%%
\subsection{Transport Fermiego-Walkera}
Konstrukcję zegara przeprowadzimy w lokalnie nierotującej bazie. W tej części
pracy przedstawimy koncepcje potrzebne do konstrukcji takiej bazy. Dokładne 
omówienie prezentowanych zagadnień można znaleźć np. 
tu~\cite{synge1960, FWframesconstruct}
Zauważmy, że dla transport równoległy wzdłuż linii geodezyjnej
przekształca wektory styczne w wektory styczne. Własność tę 
tracimy, gdy linia śwaita nie jest linią geodezyjną, czyli
gdy $A = \frac{\D u}{\d s} = 0$. Transportem, który zachowuje styczność 
wektorów do linii świata jest transport Fermiego-Walkera (FW).
Doświadczenie wskazuje, że taki transport odpowiada 
fizycznemu transportowi wektorów~\cite{Costa2015, Ashtekar20141}.
Do jego zdefiniowania posłużą nam odwzorowania $P$ i $R$.
Niech $u$ będzie jednostkowym
wektorem stycznym do linii świata $y$. Dowolny wektor $v$ możemy w punkcie 
$p\in y$ rozłożyć na składowe styczną $R(v)$ i prostopadłą $P(v)$ 
do $y$~\eqref{vnaPR}. Przestrzeń wektorów $p$ rozpada się w ten sposób 
na sumę prostą przestrzeni $\{P(v)\}$ i $\{R(v)\}$.
\begin{align}\label{vnaPR}
v = \underbrace{v - (v\cdot u) u}_{P(v)} + 
\underbrace{(v\cdot u)u}_{R(v)} = P(v) + R(v).
\end{align}

\begin{definition}Mówimy, że wektor $v$ spełnia prawo \textbf{transportu 
Fermiego-Walkera} (FW) wzdłuż linii świata $y$ jeżeli
\begin{align} 
\frac{\D_{FW} (v)}{\d s}  :=  P \left(\frac{\D P(v)}{\d s} \right) +
R \left( \frac{\D R(v)}{\d s} \right) = 0
\tag{FW} \label{FW}
\end{align}
Wyrażenie $ \frac{{\D}_{FW} }{\d s}$ nazywamy \textbf{pochodną
Fermiego-Walkera}. 
%Z tego powodu jest on w niektórych sytuacjach bardziej użyteczny. 
%Możemy więc myśleć o~pochodnej Fermiego-Walkera jako
%o~uogólnieniu pochodnej absolutnej na ruchy z~niezerowym 
%przyspieszeniem~[źródło]
%Wprowadzając takie oznacznie dostajemy warunek 
%transportu FW jako zerowanie się pochodnej Fermiego-Walkera.
\end{definition}
\begin{theorem}
Załóżmy, że $u = \dot{y}$ oraz $A=\frac{\D u}{\d s}$ 
to odpowiednio czterowektory
prędkości i przyspieszenia stoważyszone z~linią świata~$y$.
Wtedy pochodną Fermiego-Walkera możemy zapisać w postaci
\begin{align} 
\frac{\D_{FW} v}{\d s}   = 
\frac{\D v}{\d s} +
(A\cdot v) u - (u\cdot v) A.
\end{align}
Powyższa równość może służyć za definicję
pochodnej Fermiego-Walkera~\cite{synge1960} 
równoważną do tutaj przyjętej.
\end{theorem}
\begin{proof}
Obliczmy pochodne absolutne rzutów $P(v)$ oraz $R(v)$
\begin{align*}
\frac{\D P(v)}{\d s} = \frac{\D v}{\d s} - (u\cdot v) A  
- \frac{\d (u \cdot v)}{\d s} u, \quad
\frac{\D R(v)}{\d s} =\frac{\d (u \cdot v)}{\d s} u + (u\cdot v) A .
\end{align*}
Pamiętając, że $u \perp A$ mamy
\begin{align*}
P \left( \frac{\D P(v)}{\d s}  \right) 
&= \frac{\D v}{\d s} - (u\cdot v) A  
- \frac{\d (u \cdot v)}{\d s} u 
-\left(\frac{\D v}{\d s} \cdot u \right)u 
+ \frac{\d (u \cdot v)}{\d s} u =\\
&= \frac{\D v}{\d s} - (u\cdot v) A  
-\left(\frac{\D v}{\d s} \cdot u \right)u ,\\
R\left( \frac{\D R(v)}{\d s} \right) 
&=\frac{\d (u \cdot v)}{\d s} u  = 
\left(\frac{\D v}{\d s} \cdot u \right)u 
+ (A\cdot v)u .
\end{align*}
Zatem pochodna~\ref{FW} jest równa 
\begin{align} 
\frac{\D_{FW} (v)}{\d s} = P \left(\frac{\D P(v)}{\d s} \right) +
R \left( \frac{\D R(v)}{\d s} \right) = 
\frac{\D v}{\d s} +
(A\cdot v) u - (u\cdot v) A.
\end{align}
\end{proof}
W~przypadku zerowego przyspieszenia $(A\equiv 0)$ linia świata
jest linią geodezyjną, pochodna~\eqref{FW} sprowadza się 
do pochodnej absolutnej, a transport~\eqref{FW} sprowadza się 
do transportu równoległego. 

Dla dowonych wektorów $v_1$ i $v_2$ mamy $P(v_1)\perp R(v_2)$,
a więc warunek transportu~\eqref{FW} 
sprowadza się zerowania się każdego ze składników
\begin{align*}
P \left(\frac{\D P(v)}{\d s} \right)  = 0 ,\\
R \left( \frac{\D R(v)}{\d s} \right) = 0.
\end{align*}

%Wektory $P(v_1)$ oraz $R(v_2)$ są ortogonalne dla 
%dowolnych wektorów $v_1$ i $v_2$, więc możemy oddzielnie 
%rozważać zerowanie się każdego ze składników (zob. dodatek A).
%Rozpisując powyższe równości dostajemy
%\begin{align}
%\frac{\d (v \cdot u)}{\d s}=0, \tag{FW1} \label{FW1}
%\end{align}
%\begin{align}
%\left(\frac{\D (v_\perp)}{\d s}\right)_\perp = 0, 
%\text{ gdzie } v_\perp = v-(v\cdot u)u \tag{FW2} \label{FW2}
%\end{align}

%Transport równoległy wektora wzdłuż linii świata jest realizowany przez
%zerowanie się pochodnej absolutnej. 
%Niestety w ogólności transport równoległy
%nie przekształca wektorów stycznych w wektory styczne. Transport taki 
% realizuje się przez prawa transportu~\eqref{FW}. 

%Równoważna postać transportu~\eqref{FW}, to znaczy 
%warunki~\eqref{FW1} i~\eqref{FW2}, pozwala w łatwy sposób 
%skonstuować bazę, której wersory będą w owy sposób transportowane 
%wzdłuż pewnej linii świata.

\begin{definition}
\textbf{Reperem lokalnie nierotującym} nazywamy reper ruchomy poruszający
się wraz z ciałem wzdłuż jego linii świata, którego 
wersor czasowy jest styczny do linii świata (co odpowiada czteroprędkości)
i którego wersory spełniają prawo transportu~\eqref{FW}.
\end{definition}
Reper lokalnie nierotujący jest szczególnie dogodny do opisu zjawisk 
fizycznych. W granicy nierelatywistycznej odpowiada on Newtonowskiej 
koncepcji nierotującego reperu~\cite{synge1960}. Przeprowadzimy teraz
konstrukcję takiego reperu, co sprowadza się do konstrukcji 
odpowiedniej bazy $E$.

Za wersor czasowy takiej bazy możemy zawsze obrać prędkość $u$, gdyż 
jest ona unowmowanym wektorem czasowym spełniającym 
prawo transportu~\eqref{FW}
\begin{align*}
e := u = \frac{\d y}{\d s}.
\end{align*}
Dobieramy do niego wersory 
przestrzenne $e_i$, $i=1,2,3$ tak, aby otrzymana baza $E=\{ e_\mu \}$ była
 ortogonalna.  Warunek $e_i \perp e$ zapewnia, że $R(e_i)=0$. 
Zatem dodatkowym warunkiem jaki trzeba nałożyć na wersory
przestrzenne $e_i$ jest 
\begin{align*}
P \left(\frac{\D P(v)}{\d s} \right)  = 0.
\end{align*} 
Uwzględniając, że $e=u$ oraz $P(e_i)=e_i$ możemy powyższy warunek
zapisać w postaci
\begin{align}\label{FWwkw}
\frac{\D e_i }{\d s} = 
\left( \left(\frac{\D e_i }{\d s}\right) \cdot e\right) e, 
\end{align}
Przydatną własnością bazy~$E$ jest, że dany wektor 
ma w tej bazie stałe współrzędne wtedy i tylko wtedy, gdy
spełnia prawo transportu~\eqref{FW}.
Aby to pokazać wystarczy rozłożyć dany wektor w bazie 
$E$ i skorzystać z definicji transportu~\eqref{FW}.

