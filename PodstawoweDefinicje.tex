%%%%%%%%%%%%%%%%%%%%%%%%%%%%%%%%%%%%%%%%%%%%%%%%%%%%%%%%%%%%%%%%%%%%%%%%%%%%%%%
% podstawowe definicje
%%%%%%%%%%%%%%%%%%%%%%%%%%%%%%%%%%%%%%%%%%%%%%%%%%%%%%%%%%%%%%%%%%%%%%%%%%%%%%%
\section{Podstawowe pojęcia}
Modelem ogólnej teorii względności jest rozmaitość różniczkowa $M$. 
Przyjmujemy iloczyn skalarny $g$ z sygnaturą $(+,-,-,-)$.
\begin{definition}
Wektorem stycznym do krzywej $y$ nazywamy wektor $u$ taki, że
\begin{align*}
u^\mu = \frac{\d y^\mu}{\d s}.
\end{align*}
\end{definition}
\begin{definition}
Linią świata nazywamy dowolną gładką krzywą czasową, tj taką której wektor 
styczny jest wektorem czasowym. Wektor ten nazywamy czteroprędkością.
\end{definition}
\begin{definition}
Pochodną kowariantną nazywamy
\end{definition}
