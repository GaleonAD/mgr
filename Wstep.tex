%%%%%%%%%%%%%%%%%%%%%%%%%%%%%%%%%%%%%%%%%%%%%%%%%%%%%%%%%%%%%%%%%%%%%%%%%%%%%%%
% wstęp
%%%%%%%%%%%%%%%%%%%%%%%%%%%%%%%%%%%%%%%%%%%%%%%%%%%%%%%%%%%%%%%%%%%%%%%%%%%%%%%
\newpage
\section{Wstęp}
Od czasów starożytnych czas wyobrażano sobie jako jednowymiarową
rozmaitość różniczkową. 
Oczywiście pojęcie rozmaitości różniczkowej jeszcze wtedy nie istniało, 
lecz ówczesne wyobrażenia dobrze pasują do jej 
definicji~\cite{heller1993fizyka}. 
Można powiedzieć, że rozmaitości różniczkowe 
 jednowymiarowe są dwie: okrąg i prosta. 
Pozostałe jednowymiarowe rozmaitości możemy uzyskać 
poprzez rozciąganie i zginanie 
(dokładniej homeomorficzne przekształcenie) 
tychże rozmaitości.
Obserwowane zjawiska takie jak 
następujące po sobie pory roku czy też cykl faz 
Księżyca dały początek pierwszym miarom czasu. 
Prowadzi to do koncepcji czasu periodycznego, który możemy 
utożsamiać z okręgiem. 
Okrąg taki zostaje rozcięty przez zdarzenie, które występuje 
jednokrotnie. Takim zdarzeniem może być na przykład 
przyjście na świat Jezusa Chrystusa, co 
obserwujemy w postaci powszechnie 
używanego kalendarza. 
Wyobrażenie prostej wiąże się również z porządkiem, 
kolejnością zdarzeń. 
Dostatecznie mały fragment okręgu jest bardzo zbliżony do
prostej, więc w przypadku okręgu również można myśleć o
 porządku, lecz tylko w sensie lokalnym. 
W istocie wyobrażenie czasu w postaci porządku zdarzeń
pojawiło się naturalnie wcześniej niż wyobrażenie jako prosta 
w sensie ścisłym, które to zaczęło się pojawiać wraz 
z pojawieniem się ilościowego opisu przyrody~\cite{czasHeller}.


Istotna zmiana wyobrażenia czasu nastąpiła wraz z wprowadzeniem przez 
Einsteina w 1905 r. Szczególnej Teorii Względności~\cite{einstein}.
Pojawiło się mieszanie współrzędnych 
przestrzennych i czasu przy transformacji 
inercjalnych układów odniesienia.
W 1907 Minkowski nadał szczególnej teorii względności 
geometryczną postać traktując czas jako czwartą 
współrzędną~\cite{minkowski2013space}, co może 
wprowadzać pewne problemy interpretacyjne, gdyż taki 
czas płynie różnie w~różnych inercjalnych 
układach odniesienia.
Wprowadził on również pojęcie czasu własnego jako długości 
krzywej czasoprzestrzennej, po której odbywa się 
ruch, zwanej linią świata.
W tym sensie czas własny porządkuje zdarzenia czyli punkty znajdujące
się na linii świata.
Czas własny jest niezmiennikiem transformacji Lorentza, 
więc posługując się
nim nie wyróżniamy żadnego obserwatora.
%Czasoprzestrzenią Miknowskiego nazywamy
%zorientowaną czasowo czterowymiarową 
%rozmaitością różniczkową wraz z pseudoeuklidesowym 
%iloczynem skalarnym.
%Czas interpretowany jako współrzędna czasowa 
%płynie różnie w różnych wkladach odniesienia. 
%Sprawa dodatkowo komplikuje się w OTW, gdzie zakładamy 
%że czasoprzestrzeć może być zakrzywiona.
%Chcielibyśmy mierzyć czas niezależnie od wybranego układu odnisienia. 
%Odpowiednim parametrem porządkującym zdarzenia wzdłuż 
%czterowymiarowej krzywej 
%po której porusza się obserwator może być jej długość $s$. 
%Czas ten nazywamy czasem własnym, gdyż czas współrzędnościowy 
%ciała spoczywającego w 
%tym układzie odniesienia pokrywa się z $s$ (z dokładnością do 
%wyboru jednostek i punktu startowego). 
Hipoteza zegara
mówi, że istnieje zegar idealny, który odmierza czas własny 
wzdłuż swojej linii świata niezależnie od przyspieszeń (krzywizny)
jakim podlega.

W następnym rozdziale wprowadzimy pojęcia wstępne oraz omówimy
wspomnianą hipotezę. W rozdziale trzecim wprowadzamy pojęcie 
fundamentalnego relatywistycznego rotatora~\cite{star2008} oraz
zaprezentujemy model zegara idealnego. Następnie 
wprowadzamy krzywą chronometryczną, która jest motywowana 
realizacją tego modelu.
W następnych rozdziałach badamy krzywe chronometryczne
dla różnych ruchów pod kątem prawdziwości hipotezy zegara.
