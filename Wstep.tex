%%%%%%%%%%%%%%%%%%%%%%%%%%%%%%%%%%%%%%%%%%%%%%%%%%%%%%%%%%%%%%%%%%%%%%%%%%%%%%%
% wstęp
%%%%%%%%%%%%%%%%%%%%%%%%%%%%%%%%%%%%%%%%%%%%%%%%%%%%%%%%%%%%%%%%%%%%%%%%%%%%%%%
\section{Wstęp}
Od czasów starożytnych czas wyobrażano sobie jako jednowymiarowa 
rozmaitość różniczkową. 
Oczywiście pojęcie rozmaitości różniczkowej jeszcze wtedy nie istniało, 
lecz ówczesne wyobrażenia dobrze pasują do jej definicji [heller]. 
Można powiedzieć, że rozmaitości różniczkowe 
 wymiaru $1$ są dwie: okrąg i prosta. 
Pozostałe jednowymiarowe rozmaitości różniczkowe można uzyskać 
poprzez rozciąganie i zginanie (dokładniej homeomorficzne przekształcenie) 
tychże. 
Cykliczność obserwowanych zjawisk takich jak pory roku, cylk faz 
Księżyca, dały początek pierwszym miarom czasu. 
Prowadzi to do koncepcji czasu periodycznego, który możemy 
utożsamiać z okręgiem. 
Okrąg taki zostaje rozcięty przez zdarzenie, które występuje 
jednokrotnie. Takim zdarzeniem może być na przykład 
przyjście na świat Jezusa, co obserwujemy w postaci powszechnie 
używanego kalendarza. 
Wyobrażenie prostej wiąże się również z porządkiem, 
kolejnością zdarzeń. 
Dostatecznie mały fragment okręgu jest bardzo zbliżony do
prostej. Więc przypadku okręgu również można myśleć o
 porządku lecz tylko w sensie lokalnym. 
W istocie wyobrażenie czasu w postaci porządek zdarzeń
pojawiło się naturalnie wcześniej niż wyobrażenie jako prosta 
w sensie ścisłym, które to zaczęło się pojawiać wraz 
z pojawieniem się ilościowego opisu przyrodu~\cite{czas}.


Istatna zmiana wyobrażenia czasu nastąpiła wraz z wprowadzeniem przez 
Einsteina w 1905 r. szczególnej teorii względności. 
Pojawiło się mieszanie współrzędnych przestrzennych i czasu do transformacji 
inercjalnych układów odniesienia. 
Geometryczna postac nadal jej w 1907 Minkowski. 
Czasoprzestrzenią Miknowskiego nazywamy
zorientowaną czasowo czterowymiarową 
rozmaitością różniczkową wraz z pseudoeuklidesowym 
iloczynem skalarnym.
Czas interpretowany jako współrzędna czasowa 
płynie różnie w różnych wkladach odniesienia. 
Sprawa dodatkowo komplikuje się w OTW, gdzie zakładamy 
że czasoprzestrzeć może być zakrzywiona.
Chcielibyśmy mierzyć czas niezależnie od wybranego układu odnisienia. 
Odpowiednim parametrem porządkującym zdarzenia wzdłuż 
czterowymiarowej krzywej 
po której porusza się obserwator może być jej długość $s$. 
Czas ten nazywamy czasem własnym, gdyż czas współrzędnościowy 
ciała spoczywającego w 
tym układzie odniesienia pokrywa się z $s$ (z dokładnością do 
wyboru jednostek i punktu startowego). Hipoteza zegara
mówi, że istnieje zegar idealny, który odmierza czas własny 
wzdłuż swojej linii świata niezależnie od przyspieszeń 
jakim podlega.
Dokładniejszy opis hipotezy zegara znajduje się w 
rozdziale 2. W rozdziale 3 wprowadzamy pojęcie 
fundamentalnego relatywistycznego rotatora oraz 
używamy go do konstrukcji fundamentalnego zegara.
Wskazówka takiego zegara zakreśla krzywą chronometryczną.
W następnych rozdziałach badamy otrzymany model zegara
dla różnych ruchów.
