%%%%%%%%%%%%%%%%%%%%%%%%%%%%%%%%%%%%%%%%%%%%%%%%%%%%%%%%%%%%%%%%%%%%%%%%%%%%%%%
% wstęp
%%%%%%%%%%%%%%%%%%%%%%%%%%%%%%%%%%%%%%%%%%%%%%%%%%%%%%%%%%%%%%%%%%%%%%%%%%%%%%%
\newpage
\section{Wstęp}
Od czasów starożytnych czas wyobrażano sobie jako nieprzerwany zbiór chwil, 
który nauka nowożytna modeluje jednowymiarową rozmaitością różniczkową.
%Od czasów starożytnych czas wyobrażano sobie jako jednowymiarową
%rozmaitość różniczkową. 
%Oczywiście pojęcie rozmaitości różniczkowej jeszcze wtedy nie istniało, 
%lecz ówczesne wyobrażenia dobrze pasują do jej 
%definicji. 
Można powiedzieć, że jednowymiarowe rozmaitości różniczkowe 
  są dwie: okrąg i prosta. 
Pozostałe jednowymiarowe rozmaitości możemy uzyskać 
poprzez rozciąganie i zginanie 
(dokładniej homeomorficzne przekształcanie) 
tychże rozmaitości.
Obserwowane zjawiska -- takie jak 
następujące po sobie pory roku czy też cykl faz 
Księżyca -- dały początek pierwszym miarom czasu. 
To prowadzi do koncepcji czasu periodycznego, który możemy 
utożsamiać z okręgiem. 
Okrąg taki zostaje rozcięty przez zdarzenie, które występuje 
jednokrotnie. Takim zdarzeniem może być  
przyjście na świat Jezusa Chrystusa, co 
obserwujemy w postaci powszechnie 
używanego kalendarza. 
Wyobrażenie prostej wiąże się również z porządkiem, 
kolejnością zdarzeń. 
Dostatecznie mały fragment okręgu jest bardzo zbliżony do
prostej, więc w przypadku okręgu również można myśleć o
 porządku, lecz tylko w~sensie lokalnym. 
W istocie, wyobrażenie czasu w postaci porządku zdarzeń
pojawiło się naturalnie wcześniej niż wyobrażenie jako prosta 
w sensie ścisłym, które to zaczęło się pojawiać wraz 
z pojawieniem się ilościowego opisu przyrody~\cite{czasHeller}.


Istotna zmiana wyobrażenia czasu nastąpiła wraz z wprowadzeniem przez 
Einsteina w 1905 r. szczególnej \mbox{teorii} względności~\cite{einstein}.
Do transformacji czasu przy przejściu 
pomiędzy inercjalnymi układami odniesienia zostały wplecione współrzędne 
przestrzenne.
W 1907 r. Minkowski nadał szczególnej teorii względności 
geometryczną postać, traktując czas jako czwartą 
współrzędną~\cite{minkowski2013space}, co może 
wprowadzać pewne problemy interpretacyjne, gdyż taki 
czas płynie różnie w~różnych inercjalnych 
układach odniesienia.
Wprowadził on również pojęcie czasu własnego jako długości 
krzywej czasopodobnej, po której odbywa się 
ruch, zwanej linią świata.
W tym sensie czas własny porządkuje zdarzenia, 
czyli punkty znajdujące
się na linii świata.
Czas własny jest funkcjonałem określonym na~danej linii świata,
będącym niezmiennikiem 
transformacji współrzędnych, w tym lokalnych transformacji Lorentza,
więc posługując się nim, nie wyróżniamy żadnego obserwatora.
Hipoteza zegara
mówi, że istnieje zegar idealny, który odmierza czas własny 
wzdłuż swojej linii świata, niezależnie od przyspieszeń (krzywizny)
jakim podlega.

W następnym rozdziale wprowadzimy pojęcia wstępne oraz omówimy
wspomnianą hipotezę. W rozdziale trzecim wprowadzimy pojęcie 
fundamentalnego relatywistycznego rotatora~\cite{star2008} oraz
zaprezentujemy model zegara idealnego. Następnie 
wprowadzimy krzywą chronometryczną, która jest motywowana 
realizacją tego modelu. Nie stanowi ona rozwiązań równań ruchu
zegara, ale zawiera jego istotne cechy kinematyczne.
Może więc stanowić odrębny przedmiot badań, niezależnie 
od rozwiązania równań ruchu.
W następnych rozdziałach zbadamy krzywe chronometryczne
dla różnych ruchów pod kątem testowania hipotezy zegara.
