%%%%%%%%%%%%%%%%%%%%%%%%%%%%%%%%%%%%%%%%%%%%%%%%%%%%%%%%%%%%%%%%%%%%%%%%%%%%%%%
% wstęp
%%%%%%%%%%%%%%%%%%%%%%%%%%%%%%%%%%%%%%%%%%%%%%%%%%%%%%%%%%%%%%%%%%%%%%%%%%%%%%%
\section{Wstęp}
Od czasów starożytnych czas wyobrażano sobie jako jednowymiarowa 
rozmaitość różniczkową. 
Oczywiście pojęcie rozmaitości różniczkowej jeszcze wtedy nie istniało, 
lecz ówczesne wyobrażenia dobrze pasują do jej definicji [heller]. 
Rozmaitości wymiaru $1$ są dwie: okrąg i prosta. 
Pozostałe jednowymiarowe rozmaitości różniczkowe można uzyskać 
poprzez rozciąganie i zginanie (dokładniej homeomorficzne przekształcenie) 
tychże.  

Prosta wiąże się z porządkiem, kolejnością zdarzeń. 
W przypadku okręgu również można myśleć o pewnym porządku lokalnym gdyż, 
w dużym przybliżeniu, okrąg przypomina prostą. 
Okrąg wiąże się z czasem cyklicznym i wiąże się z 
obserwowaniem w przyrodzie pór roku.
...

Zmiana tego wyobrażenia nastąpiła wraz z wprowadzeniem przez 
Einsteina w 1914 szczególnej teorii względności. 
Pojawiło się mieszanie współrzędnych przestrzennych i czasu do transformacji 
inercjalnych układów odniesienia. 
Geometryczna postac nadal jej w ... Minkowski. 
STW modeluje się na czasoprzestrzeni, która jest  
zorientowaną czasowo czterowymiarową 
rozmaitością różniczkową. 
Czas interpretowany jako współrzędna czasowa 
płynie różnie w różnych wkladach odniesienia. 
Sprawa dodatkowo komplikuje się w OTW, gdzie zakładamy 
że czasoprzestrzeć może być zakrzywiona.
Chcielibyśmy mierzyć czas niezależnie od wybranego układu odnisienia. 
Odpowiednim parametrem porządkującym zdarzenia wzdłuż 
czterowymiarowej krzywej 
po której porusza się obserwator może być jej długość $s$. 
Czas ten nazywamy czasem własnym, gdyż czas współrzędnościowy 
ciała spoczywającego w 
tym układzie odniesienia pokrywa się z $s$ (z dokładnością do 
wyboru jednostek i punktu startowego). Hipoteza zegara
mówi, że istnieje zegar idealny, który odmierza czas własny 
wzdłuż swojej linii świata niezależnie od ruchu jaki wykonuje.

