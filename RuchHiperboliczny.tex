%%%%%%%%%%%%%%%%%%%%%%%%%%%%%%%%%%%%%%%%%%%%%%%%%%%%%%%%%%%%%%%%%%%%%%%%%%%%%%%
% konstrukcja reperu poruszającego 
%%%%%%%%%%%%%%%%%%%%%%%%%%%%%%%%%%%%%%%%%%%%%%%%%%%%%%%%%%%%%%%%%%%%%%%%%%%%%%%
\subsection{Ruch hiperboliczny.}
Jako pierwszy przypadek zbadamy relatywistyczny 
odpowiednik ruchu jednostajnie 
przyspieszonego.
Poniższe wyprowadzenie kształtu linii 
świata takiego ruchu można znaleźć 
w książce~\cite{trau1984}.
W układzie obserwatora inercjalnego~$\mathcal{I}$~posługującego się 
kartezjańskim układem współrzędnych rozważamy linię 
świata~$y=y(s)$~obserwatora~$\mathcal{Z}$~parametryzowaną 
czasem własnym~$s$.
Ruch ten odbywa się w jednym wymiarze przestrzennym. 
W takim przypadku ogólna postać wektora prędkości, 
po uwzględnieniu warunku 
unormowania, ma postać
\begin{align}\label{UHiper}
(u^\mu) = (\cosh \beta( s ),\ \sinh \beta( s ),\ 0,\   0),
\end{align}
gdzie $\beta( s)$ jest pewną funkcją parametryzowaną czasem własnym~$s$.
Żądamy teraz, 
aby przyspieszenie właściwe $\alpha$ 
było stałe ze względu na $s$. 
\begin{align}\nonumber
(A^\mu) = ( \dot{\beta}( s ) \sinh \beta( s ),\ 
\dot{\beta}( s ) \cosh \beta( s ),\ 0,\ 0 ),
\end{align}
\begin{align}\nonumber
\alpha = \sqrt{ - A^\mu A_\mu } = \dot{\beta}( s ).
\end{align}
Otrzymujemy równanie różniczkowe na funkcję $\beta(s)$. 
Możemy bez straty ogólności przyjąć, że $\beta(0)  = 0$. 
Wtedy rozwiązanie jest postaci
\begin{align}\nonumber
\beta( s ) = \alpha s,
\end{align}
\begin{align}\nonumber
(u^\mu) = (\cosh \alpha s ,\ \sinh \alpha s ,\ 0,\   0),
\end{align}
\begin{align}\nonumber
(A^\mu) = ( \alpha \sinh \alpha s,\ \alpha \cosh \alpha s,\ 0,\ 0 ).
\end{align}
Zatem odpowiednik ruchu jednostajnie przyspieszonego w
czasoprzestrzeni Minkowskiego to ruch opisany przez hiperbolę.
Łatwo sprawdzić, że dla małych prędkości ruch ten przechodzi 
w ruch jednostajnie przyspieszony. Ciało w takim ruchu porusza
się po linii świata
\begin{align}\label{YHiper}
(y^\mu) = \left( \frac{1}{\alpha} \sinh \alpha s,\ 
\frac{1}{\alpha} \cosh \alpha s,\ 0,\ 0 \right).
\end{align}
\\
Chcemy skonstruować reper współporuszający się z 
obserwatorem~$\mathcal{Z}$. 
W tym celu za wersor czasowy obieramy prędkość $e = u$, 
a za pierwszy z wersorów przestrzennych unormowane 
przyspieszenie $e_1 = A/\alpha$.
Wersory te uzupełniamy do bazy za pomocą wersorów kanonicznych.
Otrzymaną bazę możemy zapisać zgrabnie w postaci 
macierzy~\eqref{EHiper}.
Łatwo sprawdzić, że tak skonstruowany reper spełnia
prawo transportu~\eqref{FW}.

\begin{align}\label{EHiper}
\begin{pmatrix}
e\\
e_1\\
e_2\\
e_3
\end{pmatrix}
=
\begin{pmatrix}
\cosh \alpha s & \sinh \alpha s & 0 &   0 \\
\sinh \alpha s & \cosh \alpha s & 0 &   0 \\
0 & 0 & 1 &   0 \\
0 & 0 & 0 &   1 \\
\end{pmatrix}.
\end{align}
Możemy teraz podać równanie na fazę zegara $\phi$ 
\begin{align*}
\chi &= \pi, \quad \alpha = \text{const,} \\
\dot{\varphi} &= \pm \frac{2}{\ell} + \alpha \sin ( \varphi - \pi )
 = \pm \frac{2}{\ell} -  \alpha \sin ( \varphi ).
\end{align*}
