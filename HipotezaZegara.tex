%%%%%%%%%%%%%%%%%%%%%%%%%%%%%%%%%%%%%%%%%%%%%%%%%%%%%%%%%%%%%%%%%%%%%%%%%%%%%%%
% podstawowe definicje
%%%%%%%%%%%%%%%%%%%%%%%%%%%%%%%%%%%%%%%%%%%%%%%%%%%%%%%%%%%%%%%%%%%%%%%%%%%%%%%
\subsection{Czas własny. Hipoteza zegara.}
Spośród parametrzacji krzywych czasowych
 możemy wyróżnić tak zwaną parametryzację łukową.
Jest to parametryzacja dla której długość 
wektora styczny do krzywej ma stałą długość równą jedności.
Wyróżniony w ten sposób parametr będziemy oznaczać przez 
$s$ i nazywać czasem własnym.
Czas własny jest równy długości krzywej
\begin{align*}
s = \int 1 \d s
\end{align*}
Dla cząstki spoczywającej w danym układzie odniesienia jej prędkość
ma wyłącznie składową czasową i jest 
ona równa czasowi własnemu (z dokładnością
do stałej addytywnej i multiplikatywnej). 
Uzasadnia to nazywanie parametru $s$ czasem własnym.
\begin{definition}
Parametrem afinicznym nazywamy parametr dla którego 
prędkość w każdym punkcie krzywej ma tę samą długość.
\end{definition}
Wynika z tego natychmiast, że czas własny jest parametrem afinicznym.
Łatwo pokazać, że jeśli $s$ jest 
parametrem afinicznym, to każdy parametr afiniczny 
jest postaci $a s+b$, gdzie $a,b\in R$.
Wektor prędkości wzdłuż krzywej $y$ parametryzowanej czasem 
własnym będziemy oznaczać przez $\dot{y}$.
Mierząc czas własny nie wyrozniamy obserwatorem wiec 
postepujemy zgodnie z zasada wzglednosci Einteina. 
Do pomiaru czasu używa się zegarów.
\begin{definition}
Zegarem nazywamy dowolny układ fizyczny, w którym możemy wyodrębnić pewien
mechanizm oscylacji. Oscylacje te nazywamy częstością zegara. Czas mierzymy
ilością oscylacji.
\end{definition}
Mianem zegara idealnego określa się więc zegar mierzący czas własny
niezależnie od krzywizny krzywej po jakiej się porusza.
Ciężko wyobrazić sobie fizyczną realizację takiego zegara, jednak
zwykle zakłada się, że można konstruować coraz lepsze zegary 
tak że w granicy doskonałości otrzymamy zegar mierzący czas własny. 
Jednak jest to wyłacznie hipoteze zwana Hipotezą zegara.
Jest to hipoteza, która leży u podstaw zarówno 
szczególnej jak i ogólnej teorii względności. 
W czasoprzestrzeni minkowskiego dla 
ruchów odbywających się bez przyspieszeń 
tj po liniach geodezyjnych
hipoteza zegara jest spełniona, a 
realizacją takiego zegara może być zegar świetlny lub inaczej
zegar geometrodynamiczny (ang. geometrodynamic 
clock~\cite{ohanian2013gravitation})
W najprostrzym wydaiu składa się on z ustawionych 
vis a vis luster poruszających się po równoległych torach.
Odbijający się między nimi promień świetlny 
określa częstość pracy zegara. 
Warto wspomnieć, że hipoteza zegara 
została eksperymentalnie spradzona dla ogromnych 
przyspieszeń rzędu~$10^{19}\si{\metre\per\second^2}$~\cite{Bailey1977}.

W następnej przedstawimy fundamentalny 
relatywistyczny rotator, który może posłużyć do konstrukcji zegara 
fundamentalnego. Abstrahujemy od realizacji fizycznej. 
Nie jest ona istotna , gdyż interesuje nas granica doskonałości, 
czyli zegar, który może być
zegarem czysto matematycznym.  




