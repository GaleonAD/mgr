%%%%%%%%%%%%%%%%%%%%%%%%%%%%%%%%%%%%%%%%%%%%%%%%%%%%%%%%%%%%%%%%%%%%%%%%%%%%%%%
% podstawowe definicje
%%%%%%%%%%%%%%%%%%%%%%%%%%%%%%%%%%%%%%%%%%%%%%%%%%%%%%%%%%%%%%%%%%%%%%%%%%%%%%%
\subsection{Czas własny. Hipoteza zegara.}
Spośród parametrzacji krzywych czasowych
 możemy wyróżnić parametryzację łukową.
Jest to parametryzacja w której $y'(s)$ jest jednostkowy.
Wtedy wartość parametru $s$ jest równa długości krzywej
\begin{align}
s = \int \d s
\end{align}
Parametr ten będziemy nazywać czasem własnym i w dalszej 
części pracy będziemy oznaczać przez $s$.
UZASADNIENIE
