%%%%%%%%%%%%%%%%%%%%%%%%%%%%%%%%%%%%%%%%%%%%%%%%%%%%%%%%%%%%%%%%%%%%%%%%%%%%%%%
% podstawowe definicje
%%%%%%%%%%%%%%%%%%%%%%%%%%%%%%%%%%%%%%%%%%%%%%%%%%%%%%%%%%%%%%%%%%%%%%%%%%%%%%%
\subsection{Czas własny. Hipoteza zegara.}
Czas własny stanowi przeniesienie na grunt Ogólnej Teorii Względności 
koncepcji czasu obecnej w mechanice nierelatywistycznej.
I zasada dynamiki Newtona głosi, że istnieją ruchy będące ruchami
jednostajnymi.  W ruchu jednostajnym przemierzamy 
jednakowe odcinki drogi w jednakowych odstępach czasu. 
W ruchu swobodnym najprostszym elementem absolutnym jest 
długość linii świata t.j. suma elementarnych 
odcinków czasowych wyznaczanych przez obserwatorów
lokalnie inercjalnych. Jak pokazaliśmy długość krzywej jest
funkcjonałem pierwszego rzędu w prędkościach niezależnym 
od parametryzacji. Jeżeli przyjąc, że stan fizyczny zależy 
od położeń i ich zmiany (pochodnych) to długość krzywej jest 
jedyną możliwością na zbudowanie funkcjonału czasu
\begin{align*}
T = \int \sqrt{x'(\tau) \cdot x'(\tau)} \d \tau
\end{align*}
Spośród parametryzacji krzywych 
 możemy wyróżnić parametryzację 
dla której wektor prędkości jest 
jednostkowy. Taki parametr będziemy oznaczać przez $s$. 
Wartości tego parametru pokrywają się wartością funkcjonału czasu $T$
(z dokładnością do wyboru jednostek i chwili początkowej). 
\begin{definition}
Parametrem afinicznym nazywamy parametr dla którego 
prędkość w każdym punkcie krzywej ma tę samą długość.
\end{definition}
Wynika z tego natychmiast, że czas własny jest parametrem afinicznym.
Łatwo pokazać, że jeśli $s$ jest 
parametrem afinicznym, to każdy parametr afiniczny 
jest postaci $a s+b$, gdzie $a,b\in R$ i ta swoboda odzwierciedla
swobodę wyboru jednostki i chwili początkowej.
Wektor prędkości wzdłuż krzywej $y$ parametryzowanej czasem 
własnym będziemy oznaczać przez $\dot{y}$.
Mierząc czas własny nie wyróżniamy obserwatorem wiec 
postępujemy zgodnie z zasada względności Einteina. 

Do pomiaru czasu używa się zegarów, które możemy 
najogólniej zdefiniować następująco
\begin{definition}
Zegarem nazywamy dowolny układ fizyczny, w którym możemy wyodrębnić pewien
mechanizm oscylacji. Oscylacje te nazywamy częstością zegara. Czas mierzymy
ilością oscylacji.
\end{definition}
Mianem zegara idealnego określa się zegar mierzący czas własny
niezależnie od krzywizny krzywej po jakiej się porusza.
\textbf{Hipoteza zegara} głosi, że istnieją zegary idealne wedle 
powyższej definicji.



%Ciężko wyobrazić sobie fizyczną realizację takiego zegara, jednak
%zwykle zakłada się, że można konstruować coraz lepsze zegary 
%tak, że w granicy doskonałości otrzymamy zegar mierzący czas własny. 
%Jednak jest to wyłącznie hipoteza zwana Hipotezą zegara.
%Jest to hipoteza, która leży u podstaw zarówno 
%szczególnej jak i ogólnej teorii względności. 
%W czasoprzestrzeni Minkowskiego dla 
%ruchów odbywających się bez przyspieszeń 
%to jest po liniach geodezyjnych
%hipoteza zegara jest spełniona, a 
%realizacją takiego zegara może być zegar świetlny lub inaczej
%zegar geometrodynamiczny (ang. geometrodynamic 
%clock~\cite{ohanian2013gravitation})
%W najprostszym wydaniu składa się on z ustawionych 
%\textit{vis a vis} luster poruszających się po równoległych torach.
%Odbijający się między nimi promień świetlny 
%określa częstość pracy zegara. 
%Warto wspomnieć, że hipoteza zegara 
%została eksperymentalnie sprawdzona dla ogromnych 
%przyspieszeń rzędu~$10^{19}\si{\metre\per\second^2}$~\cite{Bailey1977}.
%W następnej części pracy przedstawimy fundamentalny 
%relatywistyczny rotator. W oparciu o niego można skonstruować model,
%który możemy uważać za realizację matematyczną zegara
%fundamentalnego. Abstrahujemy od realizacji fizycznej. 
%Nie jest ona istotna , gdyż interesuje nas granica doskonałości, 
%czyli zegar, który może być
%zegarem czysto matematycznym.  




