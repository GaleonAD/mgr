%%%%%%%%%%%%%%%%%%%%%%%%%%%%%%%%%%%%%%%%%%%%%%%%%%%%%%%%%%%%%%%%%%%%%%%%%%%%%%%
% podstawowe definicje
%%%%%%%%%%%%%%%%%%%%%%%%%%%%%%%%%%%%%%%%%%%%%%%%%%%%%%%%%%%%%%%%%%%%%%%%%%%%%%%
\subsection{Czas własny. Hipoteza zegara.}
Spośród parametrzacji krzywych czasowych
 możemy wyróżnić tak zwaną parametryzację łukową.
Jest to parametryzacja dla której długość 
wektora stycznego jest stała i równa jedności.
Wyróżniony w ten sposób parametr będziemy oznaczać przez 
$s$ i nazywać czasem własnym.
Wtedy wartość parametru $s$ jest równa długości krzywej
\begin{align}
s = \int \d s
\end{align}
\begin{definition}
Parametrem afinicznym nazywamy parametr dla którego 
prędkość w każdym punkcie krzywej ma tę samą długość.
\end{definition}
Wynika z tego natychmiast, że czas własny jest parametrem afinicznym.
Łatwo pokazać, że jeśli $s$ jest 
parametrem afinicznym, to każdy parametr afiniczny 
jest postaci $a s+b$, gdzie $a,b\in R$.
Wektor prędkości wzdłuż krzywej $y$ parametryzowanej czasem 
własnym będziemy oznaczać przez $\dot{y}$.



Nazwę tę uzasadniam......


\begin{definition}
Zegarem nazywamy dowolny układ fizyczny, w którym możemy wyodrębnić pewien
mechanizm oscylacji. Oscylacje te nazywamy częstością zegara. Czas mierzymy
ilością oscylacji.
\end{definition}
Zegar, określany mianem idealnego, powinien z pewnością posiadać następujące 
cechy:

Niekiedy zegarem idealnym nazywa się zegar, który mierzy czas własny.

Hipoteza zegara orzeka, że zegar idealny będzie mierzyć 
czas własny niezależnie od doznawanych przyspieszeń. 



Mierząc czas własny nie wyrozniamy obserwatorem wiec 
postepujemy zgodnie z zasada wzglednosci Einteina. 
Mianem zegara idealnego określa się zegar mierzący czas własny. 
Hipoteza zegara głosi, że nawet jeśli nie da się fizycznie zrealizować 
zegara idealnego, to można konstruować coraz lepsze zegary, 
tak że w granicy doskonałości otrzymamy zegar mierzący czas własny. 
Jest to hipoteza, która leży u podstaw zarówno 
szczególnej jak i ogólnej teorii względności. 
Dla ruchów odbywających się bez przyspieszeń jest to twierdzenie, a 
realizacją takiego zegara może być zegar świetlny??? 
Zawodzi on jednak gdy pojawia się przyspieszenie. ???

W następnym rozdziale przedstawimy fundamentalny 
relatywistyczny rotator, który może posłużyć do konstrukcji zegara 
fundamentalnego. Abstrahujemy od realizacji fizycznej. 
Nie jest ona istotna , gdyż interesuje nas granica doskonałości, 
czyli zegar który może być (a może powinien) 
zegarem wyłącznie matematycznym.  
