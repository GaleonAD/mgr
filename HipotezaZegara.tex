%%%%%%%%%%%%%%%%%%%%%%%%%%%%%%%%%%%%%%%%%%%%%%%%%%%%%%%%%%%%%%%%%%%%%%%%%%%%%%%
% podstawowe definicje
%%%%%%%%%%%%%%%%%%%%%%%%%%%%%%%%%%%%%%%%%%%%%%%%%%%%%%%%%%%%%%%%%%%%%%%%%%%%%%%
\subsection{Czas własny. Hipoteza zegara.}
Czas własny stanowi przeniesienie na grunt Ogólnej Teorii Względności 
koncepcji czasu obecnej w mechanice nierelatywistycznej.
Pierwsza zasada dynamiki Newtona głosi, że istnieją ruchy będące ruchami
jednostajnymi.  W~ruchu jednostajnym przemierzamy 
jednakowe odcinki drogi w jednakowych odstępach czasu. 
W~ruchu swobodnym najprostszym elementem absolutnym jest 
długość linii świata tj. suma elementarnych 
odcinków czasowych wyznaczanych przez obserwatorów
lokalnie inercjalnych. Jak pokazaliśmy długość krzywej jest
funkcjonałem pierwszego rzędu w prędkościach niezależnym 
od parametryzacji. Jeżeli przyjąć, że stan fizyczny zależy 
od położeń i ich zmiany (pochodnych) to długość krzywej jest 
jedyną możliwością na zbudowanie funkcjonału czasu
\begin{align*}
T = \int \sqrt{x'(\tau) \cdot x'(\tau)} \d \tau.
\end{align*}
Spośród parametryzacji krzywych 
 możemy wyróżnić parametryzację, 
dla której wektor prędkości jest 
jednostkowy. Taki parametr będziemy oznaczać przez $s$. 
Wartości tego parametru pokrywają się z wartością funkcjonału $T$
(z~dokładnością do wyboru jednostek i chwili początkowej). 
\begin{definition}
Parametrem afinicznym nazywamy parametr, dla którego 
prędkość w każdym punkcie krzywej ma tę samą długość.
\end{definition}
Wynika z tego natychmiast, że czas własny jest parametrem afinicznym.
Łatwo pokazać, że jeśli $s$ jest 
parametrem afinicznym, to każdy parametr afiniczny 
jest postaci $a s+b$, gdzie $a,\ b\in R$ i ta własność odzwierciedla
swobodę wyboru jednostki i chwili początkowej.
Wektor prędkości wzdłuż krzywej $y$ parametryzowanej czasem 
własnym będziemy oznaczać przez $\dot{y}$.
Mierząc czas własny nie wyróżniamy żadnego obserwatora inercjalnego,
 wiec postępujemy zgodnie z zasadą względności Einteina. 

Do pomiaru czasu używa się zegarów, które możemy 
najogólniej zdefiniować następująco:
\begin{definition}
Zegarem nazywamy dowolny układ fizyczny, w którym możemy wyodrębnić pewien
mechanizm oscylacji. Oscylacje nazywamy częstością zegara, a ich ilość jest
miarą czasu.
\end{definition}
Mianem zegara idealnego określa się zegar mierzący czas własny
niezależnie od krzywizny krzywej po jakiej się porusza.
\textbf{Hipoteza zegara} głosi, że istnieją zegary idealne wedle 
powyższej definicji.
Na gruncie Szczególnej Teorii Względności, przy założeniu ruchu
bez przyspieszeń, hipoteza zegara jest spełniona, a 
realizacją takiego zegara może być zegar świetlny lub inaczej
zegar geometrodynamiczny~\cite{ohanian2013gravitation}.
Powszechnie zakłada się, że obowiązuje 
ona również w ruchach przyspieszonych~\cite{trau1984},
jednak jest to tylko hipoteza i wymaga sprawdzenia.

Hipoteza zegara 
została eksperymentalnie sprawdzona dla ogromnych 
przyspieszeń 
rzędu~$10^{19}\si{\metre\per\second^2}$~\cite{Bailey1977}. Jednak 
ze stałych charakterystycznych dla elektronu 
można utworzyć wielkość o wymiarze przyspieszenia $m_e c^3/\hbar$, 
która jest rzędu~$10^{29}\si{\metre\per\second^2}$. Dla 
tak ogromnych przyspieszeń hipoteza zegara może się załamywać.

W rozdziale trzecim przedstawimy model zegara idealnego inspirowany
fundamentalnym relatywistycznym rotatorem. 
\textbf{Krzywa chronometryczna}, 
która daje związek między fazą zegara jako 
odpowiednio zdefiniowanym kątem związanym z ruchem 
wewnętrznym zegara, a długością krzywej 
środka masy. Należy podkreślić, że rozważana 
tu krzywa chronometryczna nie jest rozwiązaniem równań ruchu 
modelu zegara idealnego, ale jest jedynie nim motywowana.

