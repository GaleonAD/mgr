%%%%%%%%%%%%%%%%%%%%%%%%%%%%%%%%%%%%%%%%%%%%%%%%%%%%%%%%%%%%%%%%%%%%%%%%%%%%%%%
% podstawowe definicje
%%%%%%%%%%%%%%%%%%%%%%%%%%%%%%%%%%%%%%%%%%%%%%%%%%%%%%%%%%%%%%%%%%%%%%%%%%%%%%%
\subsection{Czas własny. Hipoteza zegara.}
Spośród parametrzacji krzywych czasowych
 możemy wyróżnić tak zwaną parametryzację łukową.
Jest to parametryzacja dla której długość 
wektora stycznego jest stała i równa jedności.
Wyróżniony w ten sposób parametr będziemy oznaczać przez 
$s$ i nazywać czasem własnym.
Wtedy wartość parametru $s$ jest równa długości krzywej
\begin{align}
s = \int \d s
\end{align}
\begin{definition}
Parametrem afinicznym nazywamy parametr dla którego 
prędkość w każdym punkcie krzywej ma tę samą długość.
\end{definition}
Wynika z tego natychmiast, że czas własny jest parametrem afinicznym.
Łatwo pokazać, że jeśli $s$ jest 
parametrem afinicznym, to każdy parametr afiniczny 
jest postaci $a s+b$, gdzie $a,b\in R$.
Wektor prędkości wzdłuż krzywej $y$ parametryzowanej czasem 
własnym będziemy oznaczać przez $\dot{y}$.



Nazwę tę uzasadniam......


\begin{definition}
Zegarem nazywamy dowolny układ fizyczny, w którym możemy wyodrębnić pewien
mechanizm oscylacji. Oscylacje te nazywamy częstością zegara. Czas mierzymy
ilością oscylacji.
\end{definition}
Zegar, określany mianem idealnego, powinien z pewnością posiadać następujące 
cechy:

Niekiedy zegarem idealnym nazywa się zegar, który mierzy czas własny.

Hipoteza zegara orzeka, że zegar idealny będzie mierzyć 
czas własny niezależnie od doznawanych przyspieszeń. 
