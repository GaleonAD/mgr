%%%%%%%%%%%%%%%%%%%%%%%%%%%%%%%%%%%%%%%%%%%%%%%%%%%%%%%%%%%%%%%%%%%%%%%%%%%%%%%
% Model Staruszkiewicza
%%%%%%%%%%%%%%%%%%%%%%%%%%%%%%%%%%%%%%%%%%%%%%%%%%%%%%%%%%%%%%%%%%%%%%%%%%%%%%%
\subsection{Fundametalny relatywistyczny rotator}
Za profesorem Staruszkiewiczem~\cite{star2008} wprowadzamy poniższe definicje
\begin{definition}
Relatywistyczny rotator to układ dynamiczny
 opisany przez położenie $x$ i kierunek
zerowy $k$ oraz dodatkowo dwa parametry: masę $m$ i długość $\ell$.
\end{definition}
\begin{definition}
Układ dynamiczny  nazywamy fenomenologicznym jeżeli jego niezmienniki Casimira są 
całkami ruchu. Układ dynamiczny nazywamy fundamentalnym jeżeli jego niezmienniki
Casimira są parametrami - nie zależą od wartości początkowych.
\end{definition}
W oparciu o powyższe definicje można skonstruować fundamentalny
relatywistyczny rotator. Z wielkości zawartych w definicji relatywistycznego 
rotatora możemy utworzyć bezwymiarową wielkość
\begin{align}
\xi = - \ell^2 \frac{\dot{k} \cdot \dot{k}}{ ( k \cdot \dot{x})^2 }.
\end{align}
Możemy wtedy utworzyć Lagrangian postaci
\begin{align}
L = m \sqrt{ \dot{x} \cdot \dot{x} } f( \xi ) .
\end{align}
Działanie związane z powyższym Lagrangianem 
niezmienniczy ze względu na operację cechowania
....

Niezmiennikami Casimira będą w tym przypadku $P_\mu P^\mu$ oraz
$W_\mu W^\mu$ gdzie 
\begin{align}
W_\mu = - \frac{1}{2} \varepsilon_{\mu\nu\rho\sigma}
M^{\nu\rho} P^\sigma
\end{align}


