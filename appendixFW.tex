\subsection{Równoważność warunków transportu 
Fermiego-Walkera}
\noindent
Zakładamy, że $y$ - linia świata wzdłuż której 
wykonujemy transport oraz 
\begin{align*}
u = \frac{\d y}{\d s}, \quad A = \dot{u} = \frac{\D u}{\d s}, 
\quad u\cdot A = 0,\\
R(v) = (u\cdot v) u, \quad  P(v) = v - R(v).
\end{align*}
Niech $v$ będzie wektorem zdefiniowanym wzdłóż $y$. 
Pokażemy, że następujące warunki transportu Fermiego-Walkera są 
równoważne
\begin{enumerate}
\item
\begin{align*} 
P \left(\frac{\D P(v)}{\d s} \right) +
R \left( \frac{\D R(v)}{\d s} \right) =0,
\end{align*}
\item
\begin{align*} 
 \dot{v} +
(v\cdot A) u - (v\cdot u) A =0 .
\end{align*}
\end{enumerate}
\begin{proof}
Obliczamy składniki lewej strony równości $1$.
\begin{align*}
 \frac{\D R(v)}{\d s} = 
\frac{\D ((v\cdot u)u)}{\d s} =
\frac{\d (v\cdot u)}{\d s} u + (v\cdot u) A =
(\dot{v}\cdot u) u + (v\cdot A) u + (v\cdot u) A .
\end{align*}
\begin{align*}
\frac{\D P(v)}{\d s}  = 
 \dot{v} -(\dot{v}\cdot u) u - (v\cdot A) u - (v\cdot u) A .
\end{align*}
\begin{align*}
R \left( \frac{\D R(v)}{\d s} \right) =
(\dot{v}\cdot u) u + (v\cdot A) u .
\end{align*}
\begin{align*}
P \left(\frac{\D P(v)}{\d s} \right) &= 
 \dot{v} -
(\dot{v}\cdot u) u - (v\cdot A) u - (v\cdot u) A -
( \dot{v}\cdot u - (\dot{v}\cdot u)- (v\cdot A) )u = \\
&=
 \dot{v} -
(\dot{v}\cdot u) u - (v\cdot A) u - (v\cdot u) A 
+ (v\cdot A) u = 
 \dot{v} -
(\dot{v}\cdot u) u  - (v\cdot u) A .
\end{align*}
Korzystając z powyższych obiczeń otrzymujemy ciąg równości 
\begin{align*}
P \left(\frac{\D P(v)}{\d s} \right) +
R \left( \frac{\D R(v)}{\d s} \right) = 
 \dot{v} -
(\dot{v}\cdot u) u  - (v\cdot u) A 
(\dot{v}\cdot u) u + (v\cdot A) u  = 
 \dot{v} 
 - (v\cdot u) A 
 + (v\cdot A) u  .
\end{align*}

\end{proof}


%poprzedni dowodzik nieco innego
%Pokażemy, że następujące warinki są równoważne:
%\begin{enumerate}
%\item
%\begin{align} 
%\frac{\widetilde{\D} v}{\d s}  : = \frac{\D v}{\d s} +
%(v\cdot A) u - (v\cdot u) A = 0,
%\end{align}
%\item
%\begin{align}
%\frac{\d (v \cdot u)}{\d s}=0,
%\end{align}
%\begin{align}
%\left(\frac{\D (v_\perp)}{\d s}\right)_\perp = 0, 
%\text{ gdzie } v_\perp = v-(v\cdot u)u 
%\end{align}
%\end{enumerate}
%\begin{proof}
%$1. \implies 2.$\\
%Pierwszą z równości udowodnimy mnożąc skalarnie obustronnie 
%równość $1.$ przez $u$
%\begin{align}
%0 = \frac{\D v}{\d s} \cdot u + ( v\cdot A ) (u\cdot u) - (v\cdot u)(A\cdot u)
%=\frac{\D v}{\d s} \cdot u + v\cdot \frac{\D u }{\d s} =
%\frac{\d (v \cdot u)}{\d s}
%\end{align}
%Teraz pakżemy drugą równość
%\begin{align}
%\frac{\D (v)_\perp }{\d s}=
%\frac{\D v}{\d s} - \frac{\D u}{\d s}(v\cdot u) - \frac{\d (v\cdot u)}{\d s} u
%= \frac{\D v}{\d s} - A (v \cdot u)  \stackrel{1.}{=}  -(v\cdot A)u,
%\end{align}
%\begin{align}
%\left( \frac{\D (v)_\perp }{\d s}\right)_\perp=
 %-(v\cdot A)u  + (v\cdot A) (u\cdot u) u = 0.
%\end{align}
%$2. \implies 1.$\\
%Z pierwszej równości w $2.$ mamy
%\begin{align}\label{DodatekA2to1_1}
%\frac{\D (v)_\perp}{\d s}=  \frac{\D v}{\d s} -\frac{\d (v\cdot u)}{\d s}
 %- A(v\cdot u) 
%=  \frac{\D v}{\d s}  - A(v\cdot u) 
%\end{align}
%oraz
%\begin{align}\label{DodatekA2to1_2}
%\frac{\D v}{\d s}\cdot u = - \frac{\D u}{\d s} \cdot v.
%\end{align}
%Teraz rozpisujemy drugą równość w $2.$
%\begin{align}
%0 &= \left(\frac{\D (v_\perp)}{\d s}\right)_\perp = 
%\frac{\D (v_\perp)}{\d s} - 
%\left(\frac{\D (v_\perp)}{\d s}\cdot u \right)u 
%\stackrel{\eqref{DodatekA2to1_1}}{=} \\ &
%\stackrel{\eqref{DodatekA2to1_1}}{=} 
%\frac{\D v}{\d s} - (v\cdot u)A + 
%\left( \left( \frac{\D v}{\d s} - (v\cdot u)A \right)\cdot u \right)u
%\stackrel{\eqref{DodatekA2to1_2}}{=} 
%\frac{\D v}{\d s} - (v\cdot u)A + (v\cdot A) u = 
%\frac{\widetilde{\D} v}{\d s}  .
%\end{align}
%%\begin{align}
%%\frac{\D v}{\d s} + 
%%(v\cdot A) u - (v\cdot u) A 
%%= \frac{\D v}{\d s} 
%%\end{align}
%\end{proof}
%%\subsection{Stałość współrzynników wektora w bazie B
%%lokalnie nierotującej $(e=u)$}
%%Pokażemy, że wektor ma stałe współrzynniki w bazie 
%
