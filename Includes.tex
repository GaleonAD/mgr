%%%%%%%%%%%%%%%%%%%%%%%%%%%%%%%%%%%%%%%%%%%%%%%%%%%%%%%%%%%%%%%%%%%%%%%%%%%%%%%
% global includes
%%%%%%%%%%%%%%%%%%%%%%%%%%%%%%%%%%%%%%%%%%%%%%%%%%%%%%%%%%%%%%%%%%%%%%%%%%%%%%%
\documentclass[a4paper; 11pt]{article}

% językowe
\usepackage[english,polish,german]{babel}
\usepackage{polski}		    
\usepackage[utf8]{inputenc}

% matematyczne
\usepackage{mathtools}
\usepackage{amsfonts}
\usepackage{amsmath}
\usepackage{amsthm}

\usepackage[
style=alphabetic,
sorting=none,
%
% Zastosuj styl wpisu bibliograficznego właściwy językowi publikacji.
language=autobib,
autolang=other,
% Zapisuj datę dostępu do strony WWW w formacie RRRR-MM-DD.
urldate=iso8601,
% Nie dodawaj numerów stron, na których występuje cytowanie.
backref=false,
% Podawaj ISBN.
isbn=true,
% Nie podawaj URL-i, o ile nie jest to konieczne.
url=false,
%
% Ustawienia związane z polskimi normami dla bibliografii.
maxbibnames=3,
% Jeżeli używamy BibTeXa:
backend=bibtex
]{biblatex}

\addbibresource{bib.bib}

\usepackage{graphicx}		% for graphics/pictures
\usepackage{geometry}		% for changing page layout
\usepackage{indentfirst}	% do każdego akapitu dodawane jest wcięcie
\usepackage{icomma}		    % inteligentne przecinki
\usepackage{booktabs}		% enchanced tables
\usepackage{float}		    % for pictures/graphics
\usepackage[locale=FR]{siunitx} % ładne wpisywanie wielkości i jednostek
\usepackage{verbatim}		% komentarze
\usepackage{appendix}       % załączniki

%\sisetup{per-mode=fraction}
%\sisetup{per-mode=reciprocal}
\sisetup{per-mode=symbol}
\newgeometry{tmargin=2.3cm, lmargin=1.9cm, rmargin= 1.9cm, bmargin= 2.3cm}


%%%%%%%%%%%%%%%%%%%%%%%%%%%%%%%%%%%%%%%%%%%%%%%%%%%%%%%%%%%%%%%%%%%%%%%%%%%%%%%
% local includes
%%%%%%%%%%%%%%%%%%%%%%%%%%%%%%%%%%%%%%%%%%%%%%%%%%%%%%%%%%%%%%%%%%%%%%%%%%%%%%%
%\usepackage[fixlanguage]{babelbib}
%\selectbiblanguage{polski}

\usepackage{color}          % kolorowanie tekstu
\usepackage{courier} 		% times, kurier
\usepackage{hyperref}

