%%%%%%%%%%%%%%%%%%%%%%%%%%%%%%%%%%%%%%%%%%%%%%%%%%%%%%%%%%%%%%%%%%%%%%%%%%%%%%%
% reper w ruchu po okręgu w metryce FLRW
%%%%%%%%%%%%%%%%%%%%%%%%%%%%%%%%%%%%%%%%%%%%%%%%%%%%%%%%%%%%%%%%%%%%%%%%%%%%%%%
\subsection{Ruch po okręgu względem galaktyk.}
Rozważymy teraz ponownie ruch po okręgu, z tą różnicą, że 
wiążemy obserwatora~$\mathcal{I}$ z~odległymi galaktykami w~ekspandującym 
wszechświecie. Sytuacji tej odpowiada metryka 
Friedmana-Lemaître’a-Robertsona-Walkera~(FLRW).
Dla uproszczenia zakładamy zerową krzywiznę przestrzenną.
Tensor metryczny dany jest przez
\begin{equation}\label{FLRWmetric}
(g_{\mu\nu}) = \text{diag} (1,\ -a(t)^2,\ -a(t)^2,\ -a(t)^2).
\end{equation}
Warto zauważyć, że dla $a(t) \equiv 1$ ta metryka przechodzi
w zwykłą metrykę czasoprzestrzeni Minkowskiego, a zatem 
można łatwo zweryfikować poprawność wyników.
W dalszej części tego rozdziału przyjmujemy następujące oznaczenia 
\begin{align}\nonumber
a:=a(t),\quad a':=\frac{\d a(t)}{ \d t}.
\end{align}
%\begin{equation}
%\d s^2 = \d t^2 - (a(t))^2 (\d x^2 + \d y^2 + \d z^2)
%\end{equation}
%i konsekwentnie
%\begin{align}
%(g^{\mu\nu}) = \text{diag} (1,\ -1/a^2,\ -1/a^2,\ -1/a^2).
%\end{align}
Dla tej metryki symbole Chrostofella $\Gamma^k _{ij}$ przedstawiamy
 poniżej w tablicach odpowiednio dla $k=0,\ 1,\ 2,\ 3$.
$$
\begin{array}{cccc}
\left(
    \begin{array}{cccc}
    0 & 0 & 0 & 0 \\
    0 & a a' & 0 & 0 \\
    0 & 0 & a a' & 0 \\
    0 & 0 & 0 & a a' \\
    \end{array}
    \right), &
    \left(
    \begin{array}{cccc}
    0 & \frac{a'}{a} & 0 & 0 \\
    \frac{a'}{a} & 0 & 0 & 0 \\
    0 & 0 & 0 & 0 \\
    0 & 0 & 0 & 0 \\
    \end{array}
    \right), &
    \left(
    \begin{array}{cccc}
    0 & 0 & \frac{a'}{a} & 0 \\
    0 & 0 & 0 & 0 \\
    \frac{a'}{a} & 0 & 0 & 0 \\
    0 & 0 & 0 & 0 \\
    \end{array}
    \right), &
    \left(
    \begin{array}{cccc}
    0 & 0 & 0 & \frac{a'}{a} \\
    0 & 0 & 0 & 0 \\
    0 & 0 & 0 & 0 \\
    \frac{a'}{a} & 0 & 0 & 0 \\
    \end{array}
    \right). \\
\end{array}
 $$ 
 \\
Rozważamy linie świata cząstki w ruchu po okręgu 
%$y^\mu(s) = (t(s),\ R\cos\omega t(s),\ R\sin\omega t(s),\ 0)$.
\begin{align}\nonumber
(y^\mu)= (t(s),\ R\cos\omega t(s),\ R\sin\omega t(s),\ 0).
\end{align}  
Wtedy wektory prędkości i przyspieszenia mają postać 
\begin{gather*}
(u^\mu) = 
(\gamma,\ -R\omega\gamma\sin\omega t,
\ R\omega\gamma\cos\omega t,\ 0),\\
( A^\mu )  =\left( a'a R^2\omega^2 \gamma^2(\gamma^2+1)
,\ -\frac{a'}{a}R\omega \gamma^2(\gamma^2+1) 
\sin \omega t -R\omega^2\gamma^2\cos\omega t
,\ \frac{a'}{a}R\omega \gamma^2(\gamma^2+1) 
\cos \omega t -R\omega^2\gamma^2\sin\omega t
,\ 0\right),
\end{gather*}
gdzie $\gamma= \d t/\d s=(1-a^2R^2\omega^2)^{-1/2}$.
Przyspieszenie właściwe wynosi
\begin{align}\nonumber
\alpha =\sqrt{ -A_\mu A^\mu}, \quad \text{gdzie } 
A_{\mu } A^{\mu }=-\left(\frac{a' }{a}\right)^2\left(\gamma ^2-1\right) 
\left(\gamma ^2+1\right)^2-a^2 R^2 \omega^4\gamma^4 . 
\end{align}

%Oznaczając 
%\begin{align}
%\alpha_0 = a'a R^2\omega^2 \gamma^2(\gamma^2+1), \quad
%\alpha_1 = \frac{a'}{a}R\omega \gamma^2 (\gamma^2+1), \quad
%\alpha_2 = R\omega^2\gamma^2
%\end{align}
%mamy 
%\begin{align}
%A^\mu = ( \alpha_0 ,\ -\alpha_1 \sin \omega t -\alpha_2 \cos \omega t 
%,\ -\alpha_1\cos \omega t-\alpha_2  \sin \omega t,\ 0)
%\end{align}
%\begin{align}
%A^\mu A_\mu =  (\alpha_0)^2 -a^2\left((\alpha_1)^2 
%+(\alpha_2)^2\right) = -\alpha^2 
%\end{align}
Konstrukcję reperu lokalnie nierotującego $E$ można przeprowadzić 
analogicznie do przedstawionej w poprzednim rozdziale.
 Jednakże rachunki uprościmy wykonując konstrukcję w inny sposób. 
Mianowicie stosunkowo łatwo jest 
uogólnić wersory uzyskanej wcześniej bazy~\eqref{Esimple} tak, aby 
tworzyły bazę ortonormalną w metryce~\eqref{FLRWmetric}.
Odpowiednia baza jest postaci
%Wtedy wystarczy, analogicznie jak poprzednio, 
%rozważyć obrót wersorów $e_1$ i $e_2$ w płaszczyźnie przez nie tworzonej, 
%aby uzyskać bazę $E$, której wersory spełniają prawo transportu~\eqref{FW}.
%Wspomniane uogólnienie ma postać
%\begin{align}
%\tilde{E} = 
%\begin{pmatrix}
%e\\
%e_1'\\
%e_2'\\
%e_3
%\end{pmatrix}
%=
%\begin{pmatrix}
%\gamma 			& -R\omega\gamma\sin\omega t 	
%& R\omega\gamma\cos\omega t & 0 \\
%0		 			& \frac{1}{a} \cos\omega t 					
%& \frac{1}{a} \sin\omega t 				 & 0 \\
%a R\omega\gamma  & -\frac{1}{a}\gamma\sin\omega t			
%&\frac{1}{a} \gamma\cos\omega t			 & 0 \\
%0		 			&	0											
%& 0										 & 1 \\
%\end{pmatrix}.
%\end{align}
%Po obrocie otrzymujemy następującą bazę 
\begin{align}\nonumber
E_{FLRW} = 
\begin{pmatrix}
e\\
e_1\\
e_2\\
e_3
\end{pmatrix}
=
\begin{pmatrix}
\gamma 	& -R\omega\gamma\sin\omega t 	
& R\omega\gamma\cos\omega t & 0 \\
a R\omega\gamma \sin \psi	
& \frac{1}{a} \cos\omega t \cos\psi - \frac{1}{a}\gamma \sin\omega t \sin\psi
& \frac{1}{a} \sin\omega t \cos\psi + \frac{1}{a}\gamma \cos\omega t \sin\psi
& 0 \\
a R\omega\gamma \cos \psi & -\frac{1}{a} \cos\omega t \sin\psi - 
\frac{1}{a}\gamma \sin\omega t \cos\psi
& -\frac{1}{a} \sin\omega t \sin\psi + 
\frac{1}{a}\gamma \cos\omega t \cos\psi		 & 0 \\
0&	0	& 0	& \frac{1}{a} \\
\end{pmatrix}.
\end{align}
Wersory $e$ i $e_3$ są transportowane wzdłuż linii świata 
zgodnie z prawem~\eqref{FW}. Wersory $e_1$ i $e_2$ 
zależą od kąta obrotu $\psi$. Kąt $\psi$ powinien w granicy $a\to 1$ 
przechodzić w kąt znaleziony 
dla ciała poruszającego się po~okręgu w czasoprzestrzeni Minkowskiego.
Traktujemy tę granicę jako test poprawności wyników. 
Wartość kąta $\psi$ znajdujemy 
żądając, aby wersory~$e_1$~i~$e_2$ spełniały prawo transportu~\eqref{FW}. 
Wspólne rozwiązanie dla otrzymanych równań różniczkowych 
wyrażamy przez
\begin{align}\nonumber
\psi(s) =\int_0^s -\omega \gamma(s_1)^2  \d s_1 , \quad 
\text{gdzie } \gamma(s) = (1-a(t(s))^2R^2\omega^2)^{-1/2}.
\end{align}
Mając znaleziony odpowiedni reper, obliczamy wielkości
potrzebne do równania na fazę zegara~$\varphi$.
\begin{align*}\nonumber
A\cdot e_1 &= -a'R\omega \gamma \left( \gamma^2+1 \right)\sin\psi
 + a R \omega^2\gamma^2\cos\psi ,       \\
A\cdot e_2 &= -a'R\omega \gamma \left( \gamma^2+1 \right)\cos\psi
 - a R \omega^2 \gamma^2 \sin \psi  ,     \nonumber  \\
\chi & = \text{arccos} \left( A\cdot e_1 / \alpha \right) ,
\\
\alpha & = 
a R \omega \gamma \sqrt{a'^2 \left(\gamma ^2+1\right)^2
 +   \omega^2\gamma^2} ,
\\
\dot{\varphi} &= \pm \frac{2}{\ell} + \alpha \sin (\varphi - \chi) .
\end{align*}
Rozwijamy prawą stronę równania na $\varphi$ względem $R$ dla małych promieni 
\begin{align*}
\psi &= - \omega s + O(R^2), \quad 
\alpha = a \omega  \sqrt{4 a'^2+\omega ^2} \ R +
 O(R^3), \\
\dot{\varphi} &= \pm \frac{2}{\ell} + 
a \omega  \sqrt{4 a'^2+\omega ^2} R 
\sin \left( \varphi - \text{arccos} \left( 
\frac{a \omega \cos (\omega s) 
- 2 a' \sin (\omega s)}{a\sqrt{4a'^2 + \omega^2}} 
\right) \right) + O(R^3).
\end{align*}
Dla krótkich przedziałów czasowych możemy przyjąć $a' \to 0$, co daje
\begin{align*}
\dot{\varphi} &= \pm \frac{2}{\ell} + 
a \omega ^2 R 
\sin \left( \varphi - \omega s \right) + O(R^3).
\end{align*}


