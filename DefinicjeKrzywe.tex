%%%%%%%%%%%%%%%%%%%%%%%%%%%%%%%%%%%%%%%%%%%%%%%%%%%%%%%%%%%%%%%%%%%%%%%%%%%%%%%
% krzywe i długość krzywej (reparametryzacja)
%%%%%%%%%%%%%%%%%%%%%%%%%%%%%%%%%%%%%%%%%%%%%%%%%%%%%%%%%%%%%%%%%%%%%%%%%%%%%%%
\subsection{Krzywe w czasoprzestrzeni.}
\begin{definition}
Krzywą sparametryzowaną (lub parametryzacją krzywej) nazywamy 
odwzorowanie 
% gładki homeomorfizm
% homeomorfizm
$ y_1 : I \ni \tau \to y_1(\tau) \in M$ klasy $c^\infty$, 
gdzie $I \subset R$ 
jest przedziałem otwartym (niekoniecznie skończonym).
\end{definition}
\begin{definition}
Parametrem dla krzywej sparametryzowanej $y_1$ 
nazywamy funkcję $ \tau_1: y_1(I) \ni p 
\to \tau_1(p) = y_1^{-1}( p )\in I$. 
Będziemy pisać
$\tau_1$ zamiast $\tau_1(p)$, wszędzie gdzie punkt $p$ 
wynika z kontekstu
lub jest nieistotny. 
\end{definition}
\begin{definition}
Niech $y_1: I \to M$ i $y_2:J\to M$ będą parametryzacjami.
Reparametryzacją krzywej będziemy nazywać dyfeomorfizm $f : I \to J$
klasy $C^\infty$
taki, że $y_1 = y_2 \circ f$.
\end{definition}
Jeśli dla dwóch parametryzacji istnieje reparametryzacja to 
mówimy, że są one równoważne. Można łatwo pokazać, że jest to
relacja równoważności. 
\begin{definition}
Krzywą (lub krzywą niesparametryzowaną) nazywamy klasę równoważności
parametryzacji ze wzgędu powyżej wprowadzoną relację równoważności.
Jeżeli $y$ jest krzywą, $y_1$ jej parametryzacją z parametrem $\tau_1$
to oznaczamy $y_1 =: y(\tau_1)$.
\end{definition}
\begin{definition}
Niech $(U,\xi)$ będzie mapą w $M$ oraz $p\in U$.
Wektorem stycznym do krzywej $y$ w punkcie $p$  
(lub wektorem prędkości w 
parametrze $\tau$) nazywamy wektor $y'(\tau)$ taki, że
\begin{align*}
%y' (\tau )= \frac{\d (\xi \circ y_1 )(\tau)
% }{\d \tau}\big|_{\tau(p)}
y' (\tau )= \frac{\d y_1^\mu }{\d \tau}
\end{align*}
\end{definition}
Podział wektorów wprowadzony przez tensor metryczny $g$ 
wyróżnia trzy rodzaje krzywych. 
\begin{definition}
Krzywą $y$ nazywamy krzywą czasową (zerową, przestrzenną)
jeżeli w każdym punkcie $p \in y$ wektor $y'$ jest 
wektorem czasowym
(zerowym, przestrzennym). Linią świata cząstki
jest krzywa czasowa. 
\end{definition}
Długość $S(y(\tau))$ krzywej czasowej $y$ liczymy 
ze wzoru
\begin{align}\label{dlugosc_krzywej}
S( y(\tau) ) = \int_{\tau(p_0)}^{\tau(p_1)} \sqrt{g (
y' (\tau), y' (\tau) )} \d \tau.
\end{align}
Długość krzywej nie powinna zależeć od wyboru parametryzacji.
Istotnie, długość dana wzorem~\eqref{dlugosc_krzywej}
jest niezmiennicza ze względu na reparametryzację.
Niech $\tau_1,\ \tau_2$ będą parametrami powiązanymi 
reparametryzacją $\tau_2 = f(\tau_1)$, taką że 
$f'(\tau_1) > 0$ (gdy $f'(\tau_1) <0$ rozumowanie 
przebiega analogicznie).
Wtedy stosując zmianę zmiennych całkowania dostajemy
\begin{align}
S(y(\tau_2)) = 
\int_{\tau_2(p_0)}^{\tau_2(p_1)} \sqrt{g (
y' (\tau_2), y' (\tau_2) )} \d \tau_2  = 
\int_{\tau_1(p_0)}^{\tau_1(p_1)} \sqrt{g \left(
\frac{ y' (\tau_1) }{ f'(\tau_1)}, 
\frac{ y' (\tau_1) }{ f'(\tau_1)} \right)} 
f'(\tau_1) \d \tau_1  = 
\int_{\tau_1(p_0)}^{\tau_1(p_1)} \sqrt{g (
y' (\tau_1), y' (\tau_1)  )} \d \tau_1  
=S(y(\tau_1)).
\end{align}
Stosując ten sam wzór do krzywej zerowej otrzymujemy zerową
długość.

