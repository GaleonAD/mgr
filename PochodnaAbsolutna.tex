%%%%%%%%%%%%%%%%%%%%%%%%%%%%%%%%%%%%%%%%%%%%%%%%%%%%%%%%%%%%%%%%%%%%%%%%%%%%%%%
% podstawowe definicje
%%%%%%%%%%%%%%%%%%%%%%%%%%%%%%%%%%%%%%%%%%%%%%%%%%%%%%%%%%%%%%%%%%%%%%%%%%%%%%%
\subsection{Pochodna absolutna i transport równoległy}
W trójwymiarowej przestrzeni euklidesowej 
z kartezjańskim układem współrzędnych $x^i$ 
definiuje się transport równoległy wektora $v$ w kierunku wektora 
$w$ za pomocą równania 
\begin{align*}
w^j \frac{\partial v^i}{\partial x^j} = 0
\end{align*}
To znaczy, że podczas przemieszczania wektora $v$ po krzywej, której
wektorem prędkości w parametrze $\tau$ jest wektor $w$, spełniony jest 
warunek
\begin{align*}
 \frac{\d v^i}{\d \tau} = 0
\end{align*}
Jednak w ogólności możemy mieć do czynienia z krzywoliniowym układem 
współrzędnych lub z przestrzenią zakrzywioną. W takim przypadku uogólnia
się pojęcie transportu równoległego za pomocą pochodnej kowariantnej.
\begin{definition}
Niech $x^i$ będzie układem współrzędnych. 
Pochodną kowariantną wektora $v$ w kierunku wektora $w$ nazywamy
\begin{align*}
w^\rho \nabla_\rho v^\mu = w^\nu \frac{\partial v^\mu}{\partial x^\nu}
+ \Gamma^\mu_{\nu\sigma} w^\nu v^\sigma ,
\end{align*}
gdzie $\Gamma^\mu_{\nu\sigma} $ są symbolami Christoffela, które możemy
wyznaczyć z tensora metrycznego poprzez zależność
\begin{align*}
\Gamma^\mu_{\nu\sigma} = \frac{1}{2} g^{\mu\rho} 
\left(\frac{g_{\rho\nu}}{\partial x^\sigma}+ 
\frac{g_{\rho\sigma}}{\partial x^\nu}-
\frac{g_{\nu\sigma}}{\partial x^\rho}\right)
\end{align*}
Odwzorowanie $\nabla_\rho$ nazywa się niekiedy koneksją afiniczną.
\end{definition}
\begin{definition}
Pochodną kowariantną wektora $v$ w kierunku wektora prędkości
$y'$ krzywej $y$ w parametrze $\tau$ nazywamy
pochodną absolutną wektora $v$ i oznaczamy 
\begin{align*}
\frac{\D v^\mu }{\d \tau } = \frac{\d v^\mu}{\d \tau}
+ \Gamma^\mu_{\nu\rho} v^\nu u^\rho.
\end{align*}
\end{definition}
Pochodną absolutną wektora $v$ wzdłuż krzywej $y$
parametryzowanej czasem własnym $s$
przez będziemy oznaczać przez $\D v/\d s$ lub $\dot{v}$. 
\begin{align*}
0 = \frac{\d (\dot{y} \cdot \dot{y}) }{\d s} = 
2 \frac{\D \dot{y}}{\d s} \cdot \dot{y}
\end{align*}
\begin{definition}
Linią geodezyjną (lub krzywą swobodnego spadku) nazywamy 
krzywą $y$, dla której
\begin{align*}
\frac{\D y'}{\d \tau } = b y',
\end{align*} 
gdzie $b=0$ gdy $\tau$ jest parametrem afinicznym.
\end{definition}
Będziemy w dalszej części pracy przez $u$ oraz $A$ będziemy oznaczać
prędkość oraz przyspieszenie definiowane jak następuje
\begin{align*}
u^\mu = \dot{y^\mu} = \frac{\d y^\mu}{\d s}, 
\quad A = \dot{u^\mu} =  \frac{\D u^\mu}{\d s}.
\end{align*}




