%%%%%%%%%%%%%%%%%%%%%%%%%%%%%%%%%%%%%%%%%%%%%%%%%%%%%%%%%%%%%%%%%%%%%%%%%%%%%%%
% podstawowe definicje
%%%%%%%%%%%%%%%%%%%%%%%%%%%%%%%%%%%%%%%%%%%%%%%%%%%%%%%%%%%%%%%%%%%%%%%%%%%%%%%
\subsection{Pochodna kowariantna i absolutna}
\begin{definition}
Pochodną kowariantną nazywamy ...
transport wektora wzdłuż krzywej $\gamma$ dla którego 
.... nazywamy transportem równoległym.
\end{definition}
\begin{definition}
Pochodną absolutną wektora $v$ nazywamy pochodną kowariantną w 
kierunku krzywej $y$ o wektorze stycznym $y'$ 
parametryzowanym parametrem $\tau$ i oznaczamy przez
\begin{align}
\frac{\D v^\mu }{\d \tau } = \frac{\d v^\mu}{\d \tau}
+ \Gamma^\mu_{\nu\rho} v^\nu u^\rho.
\end{align}
\end{definition}
Pochodną absolutną wektora $v$ wzdłuż krzywej $y$
parametryzowanej czasem własnym $s$
przez będziemy oznaczać przez $\dot{v}$. 
\begin{align}
0 = \frac{\d (\dot{y} \cdot \dot{y}) }{\d s} = 
2 \frac{\D \dot{y}}{\d s} \cdot \dot{y}
\end{align}
\begin{definition}
Linią geodezyjną (lub krzywą swobodnego spadku) nazywamy krzywą $y$, dla której
\begin{align}
\frac{\D y'}{\d \tau } = b y'
\end{align} 
Gdy $\tau$ jest parametrem afinicznym to $b=0$.
\end{definition}
Będziemy w dalszej części pracy przez $u$ oraz $A$ będziemy rozumieć 
prędkość oraz przyspieszenie definiowane jak następuje
\begin{align}
u^\mu = \dot{y^\mu} = \frac{\d y^\mu}{\d s}, 
\quad A = \dot{u^\mu} =  \frac{\D u^\mu}{\d s}.
\end{align}




