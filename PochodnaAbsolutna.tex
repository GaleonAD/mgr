%%%%%%%%%%%%%%%%%%%%%%%%%%%%%%%%%%%%%%%%%%%%%%%%%%%%%%%%%%%%%%%%%%%%%%%%%%%%%%%
% podstawowe definicje
%%%%%%%%%%%%%%%%%%%%%%%%%%%%%%%%%%%%%%%%%%%%%%%%%%%%%%%%%%%%%%%%%%%%%%%%%%%%%%%
\subsection{Pochodna absolutna i transport równoległy.}
W trójwymiarowej przestrzeni euklidesowej 
z kartezjańskim układem współrzędnych $x^i$
definiujemy transport równoległy wektora $v$ w kierunku wektora 
$w$ za pomocą równania 
\begin{align*}
w^j \frac{\partial v^i}{\partial x^j} = 0.
\end{align*}
To znaczy, że podczas przemieszczania wektora $v$ po krzywej, której
wektorem prędkości w parametrze $\tau$ jest wektor~$w$, spełniony jest 
warunek
\begin{align*}
 \frac{\d v^i}{\d \tau} = 0.
\end{align*}
Jednak w ogólności możemy mieć do czynienia z krzywoliniowym układem 
współrzędnych lub przestrzenią zakrzywioną. W takim przypadku uogólnia
się pojęcie transportu równoległego za pomocą pochodnej kowariantnej.
%\begin{definition}
%Niech $x^i$ będzie układem współrzędnych. 
%Pochodną kowariantną wektora $v$ w kierunku wektora $w$ nazywamy
%\begin{align*}
%w^\rho \nabla_\rho v^\mu = w^\nu \frac{\partial v^\mu}{\partial x^\nu}
%+ \Gamma^\mu_{\nu\sigma} w^\nu v^\sigma ,
%\end{align*}
%gdzie $\Gamma^\mu_{\nu\sigma} $ są symbolami Christoffela, które możemy
%wyznaczyć z tensora metrycznego poprzez zależność
%\begin{align*}
%\Gamma^\mu_{\nu\sigma} = \frac{1}{2} g^{\mu\rho} 
%\left(\frac{\partial g_{\rho\nu}}{\partial x^\sigma}+ 
%\frac{\partial g_{\rho\sigma}}{\partial x^\nu}-
%\frac{\partial g_{\nu\sigma}}{\partial x^\rho}\right)
%\end{align*}
%Odwzorowanie $\nabla_\rho$ nazywa się niekiedy koneksją afiniczną.
%\end{definition}
\begin{definition}
Niech $x^\mu$ będzie układem współrzędnych. 
Pochodną kowariantną wektora $v$ w kierunku wektora 
$w$~$=$~$\frac{\d x(\tau)}{\d \tau}$ nazywamy
\begin{align*}
w^\rho \nabla_\rho v^\mu = \frac{\d v^\mu}{\d \tau}
+ \Gamma^\mu_{\nu\sigma} w^\nu v^\sigma ,
\end{align*}
gdzie $v$ jest określony na krzywej dla której $w$ 
jest wektorem prędkości w parametrze $\tau$. Odwzorowanie 
$\nabla$ nazywamy koneksją afiniczną, a 
przez $\Gamma^\mu_{\nu\sigma} $
rozumiemy współczynniki koneksji afinicznej w układzie $x^\mu$. 
Współczynniki $\Gamma^\mu_{\nu\sigma}$ przy zmianie układu współrzędnych z $x^\mu$
na $\tilde{x}^\mu$
transformują się według reguły
\begin{align*}
\tilde{\Gamma}^\mu_{\nu\sigma} = \Gamma^\alpha_{\beta\gamma}
\frac{\partial \tilde{x}^\mu}{\partial x^\alpha }
\frac{\partial x^\beta}{\partial \tilde{x}^\nu }
\frac{\partial x^\gamma}{\partial \tilde{x}^\sigma }
+
\frac{\partial^2 x^\alpha}{\partial \tilde{x}^\nu \partial \tilde{x}^\sigma }
\frac{\partial \tilde{x}^\mu}{\partial x^\alpha }
\end{align*} 
oraz określają one całkowicie koneksję afiniczną.
W ogólnej teorii względności 
zakłada się, że koneksja jest
\begin{enumerate}
    \item zgodna z metryką ($\nabla g =0$),
    \item beztorsyjna ($\Gamma^\mu_{\nu\sigma} 
        = \Gamma^\mu_{\sigma\nu}$).
\end{enumerate} 
Wtedy koneksja jest jednoznacznie określona i nazywamy ją
 koneksją Leviego-Civity. Wówczas możemy utożsamić pojęcie 
 geodezyjnej powstałej z transportu równoległego wektora stycznego 
 oraz ekstremalnej krzywej łączącej dwa różne punkty.
 Współczynniki koneksji 
$\Gamma^\mu_{\nu\sigma}$ nazywamy są symbolami Christoffela
i możemy wyznaczyć z równości
\begin{align*}
\Gamma^\mu_{\nu\sigma} = \frac{1}{2} g^{\mu\rho} 
\left(\frac{\partial g_{\rho\nu}}{\partial x^\sigma}+ 
\frac{\partial g_{\rho\sigma}}{\partial x^\nu}-
\frac{\partial g_{\nu\sigma}}{\partial x^\rho}\right).
\end{align*}
\end{definition}
\begin{definition}
Pochodną kowariantną wektora $v$ w kierunku wektora prędkości
$y'$ krzywej $y$ w parametrze $\tau$ nazywamy
pochodną absolutną wektora $v$ i oznaczamy 
\begin{align*}
\frac{\D v^\mu }{\d \tau } = \frac{\d v^\mu}{\d \tau}
+ \Gamma^\mu_{\nu\rho} v^\nu u^\rho.
\end{align*}
\end{definition}\noindent
Pochodną absolutną wektora $v$ wzdłuż krzywej $y$
parametryzowanej czasem własnym $s$
będziemy oznaczać przez $\D v/\d s$ lub $\dot{v}$. 
Pochodna absolutna wektora o stałej długości 
jest prostopadła do niego samego.
%\begin{align*}
%0 = \frac{\d (\dot{y} \cdot \dot{y}) }{\d s} = 
%2 \frac{\D \dot{y}}{\d s} \cdot \dot{y}
%\end{align*}
\begin{definition}
Linią geodezyjną (lub krzywą swobodnego spadku) nazywamy 
krzywą $y$, dla której
\begin{align*}
\frac{\D y'}{\d \tau } = b y',
\end{align*} 
gdzie $b=0$, gdy $\tau$ jest parametrem afinicznym.
\end{definition}
W dalszej części pracy przez $u$ oraz $A$ będziemy oznaczać
prędkość oraz przyspieszenie definiowane jak następuje
\begin{align*}
u^\mu = \dot{y^\mu} = \frac{\d y^\mu}{\d s}, 
\quad A = \dot{u^\mu} =  \frac{\D u^\mu}{\d s}.
\end{align*}

